\chapter{Introduction}\label{chap:introduction}%

% Dictionaries are important
% Problem tanimi
% Neden cozmek istiyoruz
% Dictionaries take time and effort to make you can cite life of Webster or ask about Turkish lexiconners?

% talk about dictionaries
Dictionary is an extensive concept to define.
On its own, it brings forth the monolingual type into consideration.
The type where terms are presented alongside their definitions following an alphabetical order~\cite{sterkenburg_practical_2003}.
Their primary intention is to inform the user about the words~\cite{uzun_modern_2005}.
The term that precedes the definitions are defined as \emph{headword} or \emph{lemma}.
Lemmas are often the forms of the words without inflections or derivations.
The sense they convey is as comprehensive as possible, reducing the number of redundant entries in the dictionary that would otherwise be the derivatives of the unmarked form~\cite{ibrahim_usta_turkce_2006}

% We will refer to this descriptive, monolingual type as plain \emph{dictionary} as well, clarifying the specialized ones as required.

More than 7000 living languages are recognized~\footnote{\url{https://www.ethnologue.com/statistics}} and 40 of them have a sizeable presence on the internet~\footnote{\url{https://w3techs.com/technologies/history_overview/content_language/}}.
The small fraction that have is dominated by English.
% Perhaps unsurprisingly, the rest of the languages with an internet presence has wordnets. % <- fit into the narrative

While two people speaking the same language are not barred from communicating with each other, people who do not speak a common language are unable to do so.
Translation, carrying the information from one language into another bridges this gap.

Translation has been perceived as a human centric task, requiring intrinsic knowledge on both the source and the target language.
A translator has to grasp the information presented in the source language and transfer it to the target language so that when delivered, the information should convey the same meaning as it did before.

Concerning ourselves with the modern dictionaries that are still around in one form or another, we can see the immense effort, time and expertise that they require.
In a survey by \textcite{uzun_1945ten_1999}, they note that modern Turkish dictionary has taken over 6 years of work.
The project lead also changed many times.

Further research in the area contributes to more languages having access to tools that will incorporate them into the literature.
English in not the centerpiece for most natural language processing research because it is the language that can store the most information or any other linguistic property.
It's the most abundant language on web.
% TODO hubness problem
Distributions like spaCy resorts to lemmatizations like =PRON= to denote pronouns in order to collapse the senses for "I", "you", "them" etc\@.
The sense and the accompanying word for being the brother of someone's father or mother differs in Turkish\footnote{\emph{Amca} for brother of the father and \emph{Dayı} for brother of the father.}.
Yet no distinction exists for English and senses collapse together in \emph{uncle}.
This thesis is working on this subject.

Translation, information transfer from foreign languages is a valid way of enriching a language's corpora~\cite{ibrahim_usta_turkce_2006}.
\enquote{WordNet is a semantic network}\cite{fellbaum_wordnet_1998}
WordNet become synonymous with the particular lexicon design and the proper name shed its capital letters.

To our best knowledge, a succinct set of words to define a dictionary where words are presented alongside their translations and one where words from a single language are presented with their definitions with groupings for synyoynms and hypernyms are not available.
Thus, we will refer the first type as a bilingual dictionary and the second type as a monolingual dictionary.
James Somers puts down the currently available modern dictionaries by saying \enquote{The definitions are these desiccated little husks of technocratic meaningese, as if a word were no more than its coordinates in semantic space.}\cite{noauthor_youre_nodate}

The lack of WordNets for languages other than English has been tackled before %TODO reference
Yet, against %TODO how many?
senses in the English Princeton WordNet, the most comprehensive counterpart remian at ... % TODO how many
Coupled with the licensing issue which prevents scientific research from using some WordNets.
This study is constrained only to freely available WordNets.
Open Multilingual WordNet~\cite{bond_survey_2012} presents wordnets from other languages with three crucial additions; % TODO the following sentence is plagiarism
the data is normalized, aligned with English Princeton WordNet and they are all accessible from a single source~\footnote{\url{http://compling.hss.ntu.edu.sg/omw/}}.
With alignment information at hand, we have a way to create a golden corpora that we assume to be perfectly aligned.
Among the %TODO how many
wordnets available on Open Multilingual WordNet, only %TODO
has definition or gloss information built in.
Given this thesis will only investigate the ability to map senses using definitinos of the sense, we used the subset of Albanian~\cite{ruci_current_2008}, Bulgarian~\cite{simov_constructing_2010}, Greek~\cite{stamou_exploring_2004}, Italian~\cite{pianta_multiwordnet_2002}, Slovenian~\cite{fiser_slownet_2012} and Romanian~\cite{tufis_romanian_2008}.
\begin{table*}[!hbp]
    \begin{center}
        \caption{Summary of the Wordnets used.}\label{tab:summary_table}
        \begin{tabular}{llrr}
            \toprule%
            \textbf{Name of the Project} & \textbf{Language} & \textbf{Number of Definitions} & \textbf{Words Per Definition} \\
            \midrule%
            Albanet & Albanian & 4681 & 11.75 \\
            BulTreeBank WordNet & Bulgarian & 4959 & 12.71 \\
            % DanNet & Danish    & 784                   & 8.63                           \\
            Greek Wordnet & Greek & 18136 & 11.24 \\
            ItalWordnet & Italian & 12688 & 7.33 \\
            Romanian Wordnet & Romanian & 58754 & 9.98 \\
            SloWNet & Slovenian & 3144 & 12.68 \\
            \bottomrule %
        \end{tabular}
    \end{center}
\end{table*}

