\chapter{Introduction}\label{chap:introduction}%
% Problem tanimi
% Neden cozmek istiyoruz
% Dictionaries take time and effort to make you can cite life of Webster or ask about Turkish lexiconners?
% Dictionaries are important
% talk about dictionaries
\section{Dictionaries}%
\label{sec:dictionaries}
Dictionary is a broad concept to define since lexicographers prepare them for specific use cases.
For instance, bilingual dictionaries present words alongside their translations in the target language.
Domain specific dictionaries can list technical terms that target people who are familiar with the terminology.
Yet, the term \emph{dictionary} on its own brings forth the monolingual type into consideration.
This type of dictionary presents words alongside their definitions following an alphabetical order~\cite{sterkenburg_practical_2003}.
Their primary intention is to inform the user about the words~\cite{uzun_modern_2005}.
Some words are polysemous, sharing the same spelling and having related, often derivative meanings.
These different senses are clarified through multiple definition entries.
Homonymous words have different distinct meanings while coincidentally having identical spellings.
They are often noted through a different block of descriptions.
The terms that precede the entries are called \emph{headword} or \emph{lemma}.
Usually, lemmas are the form of a word without inflections or derivations.
The sense they convey is as comprehensive as possible, reducing the number of otherwise redundant entries that would be the derivatives of the unmarked form~\cite{ibrahim_usta_turkce_2006}.

% from dictionaries to multilingual web
Dictionaries take an immense amount of time and expertise to prepare.
We can list the examples after narrowing our scope down to the dictionaries that people still use today.
A survey by \textcite{uzun_1945ten_1999} notes that the first instalment of the modern Turkish dictionary, led by a team of experts, has taken over 6 years to prepare.
\textcite{kendall_forgotten_2011} talks about how Noah Webster, the writer of the \emph{An American Dictionary of the English Language} had to mortgage off his home in order to finish his project which took over 26 years.
Early lexicographers had to work for years in order to collect material to prepare a corpus~\cite{uzun_1945ten_1999}.
This endeavour was necessary since a corpus is crucial to create the vocabulary of a language, otherwise known as a \emph{lexicon}.

The internet radically changed the way researchers aggreate the data.
The advancements in digital storage technology allowed the data to be persistent.
Improvements in networking ensured that people can share the volume of it between themselves.
With the popularization of the social media, the internet generates everyday conversations that are useful in natural language applications.
On the other hand, efforts on open, collaborative, web based encyclopedias generate structured, multilingual data.
Once cumbersome task of corpus attainment is now akin to web crawling.

With the digitized data, it was only natural for dictionaries to go digital as well.
% start wordnets
\section{WordNet}% not sold on having wordnet as a section
\label{sec:wordnet}
WordNet~\cite{fellbaum_wordnet_1998-1} is a lexical database.
In the context of dictionaries, we have emphasized lemmas and definitions.
WordNet can show relationships like hyponymym hierarchy transitive or meronymym part-whole relation, only possible with an electronic resource.

\enquote{WordNet is a semantic network}\cite{fellbaum_wordnet_1998-1}

Authorities list more than 7000 living languages\footnote{\url{https://www.ethnologue.com/statistics}} with only 40\footnote{\url{https://w3techs.com/technologies/history_overview/content_language/}} of them having a sizeable presence on the internet.
Nevertheless, among this small fraction, English is the dominant language of the web.

Translation, information transfer from foreign languages is a valid way of enriching a language's corpora~\cite{ibrahim_usta_turkce_2006}.

We should note that the languages of the wordnets used in the thesis are all present in the 40 languages that have a significant presence on the internet that we have mentioned before.
Linking wordnets leads to more research~\cite{sagot_building_2008}.

\textcite{fellbaum_semantic_1998} defines the correct terminology that we abide for the thesis; \enquote{As WordNet became synonymous with a particular kind of lexicon design, the proper name shed its capital letters and became a common designator for semantic networks of natural languages}.
Hence \emph{WordNet} refers to English Princeton WordNet, while wordnets created for other languages are not stylized.

Further research in the area contributes to more languages having access to tools that will incorporate them into the literature.
English in not the centrepiece for most natural language processing research because it is the language that can store the most information or any other linguistic advantage.
It's the most abundant language on web.
Distributions like spaCy resorts to lemmatizations like =PRON= to denote pronouns in order to collapse the senses for "I", "you", "them" etc.\@.
% might be too cheesy
The sense and the accompanying word for being the brother of someone's father or mother differs in Turkish.\footnote{\emph{Amca} for brother of the father and \emph{Dayı} for brother of the father.}
Yet no distinction exists for English and senses collapse together in \emph{uncle}.
% \might be too cheesy


James Somers puts down the modern dictionaries by saying \enquote{The definitions are these desiccated little husks of technocratic meaningese, as if a word were no more than its coordinates in semantic space.}\cite{somers_youre_2014}.

The work on wordnets for languages other than English has been under way since the early days of the predecessor.

Research has tackled the issue of lack of wordnets for languages besides English. %TODO reference

Yet, against %TODO how many?
senses in the English Princeton WordNet, the most comprehensive counterpart remain at ... % TODO how many
Coupled with the licensing issue which prevents scientific research from using some WordNets.
We have constrained this study to use only the freely available wordnets.
Open Multilingual WordNet~\cite{bond_survey_2012} presents wordnets from other languages with three crucial additions; % TODO the following sentence is plagiarism
They have normalized the data, aligned with English Princeton WordNet and they are all accessible from a single source.\footnote{\url{http://compling.hss.ntu.edu.sg/omw/}}
With alignment information at hand, we have a way to create a golden corpora that we assume to be perfectly aligned.
Among the 34 wordnets available on Open Multilingual WordNet, only 6 of them have gloss information available.
Given this thesis will only investigate the ability to map senses using definitions of the sense, we used the subset of Albanian~\cite{ruci_current_2008}, Bulgarian~\cite{simov_constructing_2010}, Greek~\cite{stamou_exploring_2004}, Italian~\cite{pianta_multiwordnet_2002}, Slovenian~\cite{fiser_slownet_2012} and Romanian~\cite{tufis_romanian_2008}.
\begin{table*}[!hbp]
    \begin{center}
        \caption{Summary of the Wordnets used.}\label{tab:summary_table}
        \begin{tabular}{llrr}
            \toprule%
            \textbf{Name of the Project} & \textbf{Language} & \textbf{Number of Definitions} & \textbf{Words Per Definition} \\
            \midrule%
            Albanet & Albanian & 4681 & 11.75 \\
            BulTreeBank WordNet & Bulgarian & 4959 & 12.71 \\
            % DanNet & Danish    & 784                   & 8.63                           \\
            Greek Wordnet & Greek & 18136 & 11.24 \\
            ItalWordnet & Italian & 12688 & 7.33 \\
            Romanian Wordnet & Romanian & 58754 & 9.98 \\
            SloWNet & Slovenian & 3144 & 12.68 \\
            \bottomrule %
        \end{tabular}
    \end{center}
\end{table*}

