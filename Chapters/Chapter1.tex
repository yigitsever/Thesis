\chapter{Introduction}\label{chap:introduction}%
% Problem tanimi
% Neden cozmek istiyoruz
% Dictionaries take time and effort to make you can cite life of Webster or ask about Turkish lexiconners?
% Dictionaries are important
% talk about dictionaries
\section{Dictionaries}%
\label{sec:dictionaries}
Dictionary is an extensive concept to define.
Lexicographers prepare them for specific use cases.
Bilingual dictionaries present words alongside their translations in the target language.
Domain specific dictionaries can list technical terms that target people who are familiar with the terminology.
Yet, the term \emph{dictionary} on its own brings forth the monolingual type into consideration.
This type of dictionary presents words alongside their definitions following an alphabetical order~\cite{sterkenburg_practical_2003}.
Their primary intention is to inform the user about the words~\cite{uzun_modern_2005}.
The terms that precede the entries are called \emph{headword} or \emph{lemma}.
Usually, lemmas are the form of a word without inflections or derivations.
The sense they convey is as comprehensive as possible, reducing the number of otherwise redundant entries that would be the derivatives of the unmarked form~\cite{ibrahim_usta_turkce_2006}.
% happy with this paragraph

% from dictionaries to multilingual web
Concerning ourselves with the modern dictionaries that are still around in one form or another, we can see the immense effort, time and expertise that they require.
A survey by \textcite{uzun_1945ten_1999} notes that the first instalment of the modern Turkish dictionary, led by a team of experts, has taken over 6 years.
\textcite{kendall_forgotten_2011} talks about how Noah Webster, the writer of the \emph{An American Dictionary of the English Language} had to mortgage off his home in order to finish his project which took over 26 years.

Early lexicographers had to spend years of research in order to collect material to form a corpus~\cite{uzun_1945ten_1999}.
This endeavour was necessary since a corpus is crucial for creating the vocabulary of a language, otherwise known as a \emph{lexicon}.
With the advancements in the internet, storage and the popularization of the social media, once difficult task of corpus attainment is now akin to crawling the web.
Authorities list more than 7000 living languages~\footnote{\url{https://www.ethnologue.com/statistics}} and only 40 of them have a sizeable presence on the internet~\footnote{\url{https://w3techs.com/technologies/history_overview/content_language/}}.
Nevertheless, among this small fraction, English is the dominant language of the web.

While two people speaking the same language are not barred from communicating with each other, people who do not speak a common language are unable to do so.
Translation, carrying the information from one language into another bridges this gap.
Translation has been perceived as a human centric task, requiring intrinsic knowledge on both the source and the target language.
A translator has to grasp the information presented in the source language and transfer it to the target language so that when delivered, the information should convey the same meaning.
Translation, information transfer from foreign languages is a valid way of enriching a language's corpora~\cite{ibrahim_usta_turkce_2006}.

It should be noted that the languages of the wordnets used in the thesis are all present in the 40 languages that have a sizeable presence on the internet that we have mentioned before.
Linking wordnets leads to more research~\cite{sagot_building_2008}.

\textcite{fellbaum_semantic_1998} defines the correct terminology that we abide for the thesis; \enquote{As WordNet became synonymous with a particular kind of lexicon design, the proper name shed its capital letters and became a common designator for semantic networks of natural languages}.
Hence \emph{WordNet} refers to English Princeton WordNet, while wordnets created for other languages are not stylized.

Further research in the area contributes to more languages having access to tools that will incorporate them into the literature.
English in not the centrepiece for most natural language processing research because it is the language that can store the most information or any other linguistic advantage.
It's the most abundant language on web.
Distributions like spaCy resorts to lemmatizations like =PRON= to denote pronouns in order to collapse the senses for "I", "you", "them" etc.\@.
% might be too cheesy
The sense and the accompanying word for being the brother of someone's father or mother differs in Turkish\footnote{\emph{Amca} for brother of the father and \emph{Dayı} for brother of the father.}.
Yet no distinction exists for English and senses collapse together in \emph{uncle}.
% \might be too cheesy
\enquote{WordNet is a semantic network}\cite{fellbaum_wordnet_1998-1}

James Somers puts down the modern dictionaries by saying \enquote{The definitions are these desiccated little husks of technocratic meaningese, as if a word were no more than its coordinates in semantic space.}\cite{somers_youre_2014}.

The work on wordnets for languages other than English has been under way since the early days of the predecessor.

The lack of wordnets for languages other than English has been tackled before %TODO reference
Yet, against %TODO how many?
senses in the English Princeton WordNet, the most comprehensive counterpart remain at ... % TODO how many
Coupled with the licensing issue which prevents scientific research from using some WordNets.
This study is constrained only to freely available wordnets.
Open Multilingual WordNet~\cite{bond_survey_2012} presents wordnets from other languages with three crucial additions; % TODO the following sentence is plagiarism
the data is normalized, aligned with English Princeton WordNet and they are all accessible from a single source~\footnote{\url{http://compling.hss.ntu.edu.sg/omw/}}.
With alignment information at hand, we have a way to create a golden corpora that we assume to be perfectly aligned.
Among the 34 wordnets available on Open Multilingual WordNet, only 6 of them have gloss information available.
Given this thesis will only investigate the ability to map senses using definitions of the sense, we used the subset of Albanian~\cite{ruci_current_2008}, Bulgarian~\cite{simov_constructing_2010}, Greek~\cite{stamou_exploring_2004}, Italian~\cite{pianta_multiwordnet_2002}, Slovenian~\cite{fiser_slownet_2012} and Romanian~\cite{tufis_romanian_2008}.
\begin{table*}[!hbp]
    \begin{center}
        \caption{Summary of the Wordnets used.}\label{tab:summary_table}
        \begin{tabular}{llrr}
            \toprule%
            \textbf{Name of the Project} & \textbf{Language} & \textbf{Number of Definitions} & \textbf{Words Per Definition} \\
            \midrule%
            Albanet & Albanian & 4681 & 11.75 \\
            BulTreeBank WordNet & Bulgarian & 4959 & 12.71 \\
            % DanNet & Danish    & 784                   & 8.63                           \\
            Greek Wordnet & Greek & 18136 & 11.24 \\
            ItalWordnet & Italian & 12688 & 7.33 \\
            Romanian Wordnet & Romanian & 58754 & 9.98 \\
            SloWNet & Slovenian & 3144 & 12.68 \\
            \bottomrule %
        \end{tabular}
    \end{center}
\end{table*}

