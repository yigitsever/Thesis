% Chapter Template
\chapter{Method}
\label{chap:method}
%%% TODO, muhtemelen degisecek

%%% ROUGH FIRST DRAFT
\section{Machine Translation}

The first method we have investigated works naively by translating the target language's corpora to English using Google Cloud API\@. %TODO ref
As before, we have created a baseline/golden/basis aligned corpora where English WordNet definitions are aligned to the translated target language definitions.
Casting the task to monolingual retrieval, we can establish a baseline using \tfidf{} retrieval.
We have chosen \tfidf{} as to ask if the task at hand can be solved by naive tools.
In order to get \tfidf{} scores of the documents, first a term-document matrix is created.
Documents being definitions and with an average of 10.62 words per definition, the resulting matrix is parse.
In a \tfidf{} matrix, for an entry in the matrix $w_{i,j}$, we can give the formula for it as:
\begin{equation*}
    \varB{tf_{w,d}-idf_{w}} = {\sum_{w' \in d}{f_{w',d}}} \cdot \log \frac {N} {df_w}
\end{equation*}
Such that term $w_{i,j}$ depicts the importance of term $t$ with relation to its general importance throughout the corpus.
Now we can define the similarity between the documents as the cosine similarity between their \tfidf{} vectors.
For the row $w_t$ and $w_p$, cosine similarity between definitions $t, p$ is
\begin{equation*}
    cos(\theta) =
\end{equation*}

Definitions are then separated into queries and corpora.
Query definitions is then matched up against every definition in the corpora and the ten documents that are closest in terms of cosine similarity is retrieved.
Within the retrieved documents, if the document with the matched sense id is retrieved in the first result, this is taken as a hit at 1.
Mean Reciprocal Rank is also calculated in order to show the success of a retrieval scenario.

Where monolingual retrieval falls short, we leveraged the power of word embeddings to capture the semantic information of the words.
A famous example for the inadequacy of \tfidf{} is illustrated by~\cite{kusner_word_2015-2}.
For two snippets of text; \emph{Obama speaks to the media in Illinois} and \emph{The President greets the press in Chicago} Kusner argues that while they convey the same information, they would be near orthogonal in a bag of words setting.
Yet before moving forward with WMD, we wanted to test sentence embeddings.

%%% CITATION
\textcite{edilson_a._correa_nilc-usp_2017} used sentence embeddings that were tailored for short text.
Their work was on Twitter where the need for word embeddings to capture the essence of the text is crucial given the low amount of data packed in a Twitter document or a tweet.
For our purposes, we used sentence embeddings as described in their implementation;
% For sentence embeddings, first a \tfidf{} matrix is constructed.
Then, with the term-embedding matrix at hand, we have calculated sentence embeddings using;
\begin{equation}
    S_{\text{emb}}(S) = \sum_{w_{i} \in S} \varB{tf_{w_{i},S}-idf_{w_i}} \cdot Emb_{w}(w_{i})
\end{equation}
Every word that makes up a definition is scaled by its vector in ${\rm I\!R}^n$, then concatenated to form sentence embeddings on ${\rm I\!R}^n$.

Given the N vectors from source and target language, we hypothesize that there exists a matching where every source definition vector is perfectly mapped to one target vector.
Given that this problem naively iterates over $N!$ matchings, we have looked into an algorithm.

%%% TODO lapjv %%%


