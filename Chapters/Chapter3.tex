% Chapter Template
\chapter{Unsupervised Matching}%
\label{chap:unsupervised_matching}

%%% ROUGH FIRST DRAFT
\section{Machine Translation}

The first method we have investigated works naively by translating the target language's corpora to English using Google Cloud API\@. %TODO ref
As before, we have created a baseline/golden/basis aligned corpora where English WordNet definitions are aligned to the translated target language definitions.
Casting the task to monolingual retrieval, we can establish a baseline using \tfidf{} retrieval.
We have chosen \tfidf{} as to ask if the task at hand can be solved by naive tools.
In order to get \tfidf{} scores of the documents, first a term-document matrix is created.
Documents being definitions and with an average of 10.62 words per definition, the resulting matrix is parse.
In a \tfidf{} matrix, for an entry in the matrix $w_{i,j}$, we can give the formula for it as:
\begin{equation*}
    \varB{tf_{w,d}-idf_{w}} = {\sum_{w' \in d}{f_{w',d}}} \cdot \log \frac {N} {df_w}
\end{equation*}
Such that term $w_{i,j}$ depicts the importance of term $t$ with relation to its general importance throughout the corpus.
Now we can define the similarity between the documents as the cosine similarity between their \tfidf{} vectors.
For the row $w_t$ and $w_p$, cosine similarity between definitions $t, p$ is
\begin{equation*}
    \cos(\theta) =
\end{equation*}

Definitions are then separated into queries and corpora.
Query definitions is then matched up against every definition in the corpora and the ten documents that are closest in terms of cosine similarity is retrieved.
Within the retrieved documents, if the document with the matched sense id is retrieved in the first result, this is taken as a hit at 1.
Mean Reciprocal Rank is also calculated in order to show the success of a retrieval scenario.

Where monolingual retrieval falls short, we leveraged the power of word embeddings to capture the semantic information of the words.
A famous example for the inadequacy of \tfidf{} is illustrated by~\cite{kusner_word_2015}.
For two snippets of text; \emph{Obama speaks to the media in Illinois} and \emph{The President greets the press in Chicago} Kusner argues that while they convey the same information, they would be near orthogonal in a bag of words setting.
Yet before moving forward with WMD, we wanted to test sentence embeddings.

\section{Linear Assignment Using Sentence Embeddings}%
\label{sec:linear_assignment_using_sentence_embeddings}

Using word embeddings to obtain embeddings for longer pieces of text has been studied with implementations like doc2vec~\cite{le_distributed_2014} that builds upon the word2vec~\cite{mikolov_distributed_2013} model in order to learn paragraph embeddings.
However, there is an assumption of a continuous text for the given model.
When the text that we would like to show on a latent space is not part of a longer piece of text but \emph{discrete} pieces, that presumption does not hold.
With the dictionary definitions, we have such a case.
Our dictionary definitions are comprised of 10 to 11 words and there is no relation from one distinct dictionary definition to another.
% TODO reference the wordnet statistics table when you decide on where to put it
In other words, they are not continuous.
One other case where a similar situation occur is \emph{twitter}.
\emph{Tweets} are short pieces of text due to the 280 character constraint imposed by the platform.
With such short pieces of text, instead of paragraph embeddings, we can talk about \emph{sentence embeddings}.
A sentence embedding model should ideally capture the collective meaning of the short text where every word is potentially informative.

\textcite{zhao_ecnu_2015} used two approaches for SemEval-2015 Task 2: Semantic Textual Similarity~\footnote{\url{http://alt.qcri.org/semeval2015/task2/}}.
First, for a sentence $S = (w_{1}, w_{2}, \dots, w_{s})$ where the length of the presumably small sentence is $|S| = s$ and the word embedding of a $w_t$ is $v_t$;
\begin{itemize}
    \item They summed up the word embeddings of the sentence $\sum_{t \in S}v_{t}$
    \item Used information content~\cite{saric_takelab_2012} to weigh each word's LSA vector $\sum_{t \in S} I(w_t) v_{t}$
\end{itemize}
Both approaches results in a vector that is in the same dimensions $R^{d}$ as the original word representations.

\textcite{edilson_a._correa_nilc-usp_2017} expanded upon this simple yet effective idea to tackle the SemEval-2017 Task 4\footnote{\url{http://alt.qcri.org/semeval2017/task4}}, Sentiment Analysis in Twitter.
In order to acquire embeddings that represented \emph{tweets}, they weighed the word embeddings that made up a tweet; $\text{tweet}_i = (w_{i1}, w_{i2}, \dots, w_{im})$ with the \tfidf{} weights.
For the \tfidf{} calculation, they cast individual weights as documents so that term frequency become the term count in a single tweet while document frequency become the number of tweets the term $w_t$ occurs.

We have mentioned that our dictionary definitions are not continuous.
Yet, we advocate using \tfidf{} weights to weigh our word embeddings to get sentence embeddings.
In order to clarify, let us present Table~\ref{tab:en_it_examples}.

\noindent\fbox{%
    \parbox{\textwidth}{%
        % \caption{Example English Princeton WordNet definitions}
        turn red, as if in embarrassment or shame \\
        a feeling of extreme joy \\
        a person who charms others (usually by personal attractiveness) \\
        so as to appear worn and threadbare or dilapidated \\
        a large indefinite number \\
        distributed in portions (often equal) on the basis of a plan or purpose \\
        a lengthy rebuke
    }%
}%
% TODO maybe a fbox with more line width but captions don't work

\begin{table}
    \centering
    \caption{Some definitions from English Princeton WordNet}%
    \label{tab:en_it_examples}
    \begin{tabular}{l}
        \toprule
        turn red, as if in embarrassment or shame \\
        a feeling of extreme joy \\
        a person who charms others (usually by personal attractiveness) \\
        so as to appear worn and threadbare or dilapidated \\
        a large indefinite number \\
        distributed in portions (often equal) on the basis of a plan or purpose \\
        a lengthy rebuke \\
        \bottomrule
    \end{tabular}
\end{table}
% TODO this table is ugly can we do better?

% For sentence embeddings, first a \tfidf{} matrix is constructed.
For the \tfidf{} calculations, we followed a similar approach.
The term frequency is the raw count of a term in a dictionary definition.
While the document frequency is the number of dictionary definitions where $w_t$ occurs.

Then, with the term-embedding matrix at hand, we have calculated definition embeddings using;
\begin{equation}
    S_{\text{emb}}(S) = \sum_{w_{i} \in S} \varB{tf_{w_{i},S}-idf_{w_i}} \cdot Emb_{w}(w_{i})
\end{equation}
Every word that makes up a definition is scaled by its vector in ${\rm I\!R}^n$, then concatenated to form sentence embeddings on ${\rm I\!R}^n$.

Given the N vectors from source and target language, we hypothesize that there exists a matching where every source definition vector is perfectly mapped to one target vector.
Given that this problem naively iterates over $N!$ matchings, we have looked into an algorithm.
% TODO are you sure about N!

%%% TODO lapjv %%%
% https://blog.sourced.tech/post/lapjv/
%

