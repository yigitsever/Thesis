% Chapter Template
\chapter{Unsupervised Matching}%
\label{chap:unsupervised_matching}

%%% ROUGH FIRST DRAFT

\section{Linear Assignment Using Sentence Embeddings}%
\label{sec:linear_assignment_using_sentence_embeddings}
% TODO down below two tracks for sentence embeddings will crash, print out and clean up
From word embeddings that represented single tokens, research also tackled the representation of longer pieces of text like documents, paragraphs and sentences.
Implementations like \textcite{le_distributed_2014} extended the skip-gram idea of \textcite{mikolov_distributed_2013} to represent \emph{paragrahs} as feature vectors and used them to predict surrounding paragraphs in the text.
While approaches like \textcite{kiros_skip-thought_2015} trained an encoder that constructed surrounding sentences to learn to learn sentence representations.
% TODO This can go to chapter2 and get expanded if need be
However, there is an assumption of a continuous text for the given models.
When the text that we would like to show on a latent space is not part of a longer piece of text but \emph{discrete} pieces, that assumption does not hold.
With the dictionary definitions, we have such a case.
Our dictionary definitions are comprised of 10 to 11 words and there is no relation from one distinct dictionary definition to another.
% TODO reference the wordnet statistics table when you decide on where to put it
In other words, they are not continuous.
One other case where a similar situation occurs is \emph{twitter}.
\emph{Tweets} are short pieces of text due to the 280 character constraint imposed by the platform.
With such short pieces of text, instead of paragraph embeddings, we can talk about \emph{sentence embeddings}.
A sentence embedding model should ideally capture the collective meaning of the short text where every word is potentially informative.

Thankfully, \textcite{wieting_towards_2015} studied sentence embeddings and reported that averaging word embeddings that make up a sentence to get sentence embeddings is a valid and surprisingly effective approach.
\textcite{arora_simple_2016} built upon that idea and presented that weighed average of word vectors perform so well, they decided to call it; \citetitle{arora_simple_2016}.
In the suggested approach, word embeddings are weighed with a scale called \emph{smooth inverse frequency}.
Smooth inverse frequency is reported as $a/(a + p(w))$ where $a$ is a parameter and $p(w)$ is the word frequency.
The authors point out that the metric is similar to \tfidf{} weighting scheme if \enquote{one treats a \enquote{sentence} as a \enquote{document} and make the reasonable assumption that the sentence doesn't typically contain repeated words}.
These assumptions hold for us so we scaled our word embeddings using \tfidf{} weights to get sentence embeddings.

Parallel to
\textcite{zhao_ecnu_2015} used two approaches for SemEval-2015 Task 2: Semantic Textual Similarity~\footnote{\url{http://alt.qcri.org/semeval2015/task2/}}.
First, for a sentence $S = (w_{1}, w_{2}, \dots, w_{s})$ where the length of the presumably small sentence is $|S| = s$ and the word embedding of a $w_t$ is $v_t$;
\begin{itemize}
    \item They summed up the word embeddings of the sentence $\sum_{t \in S}v_{t}$
    \item Used information content~\cite{saric_takelab_2012} to weigh each word's LSA vector $\sum_{t \in S} I(w_t) v_{t}$
\end{itemize}
Both approaches results in a vector that is in the same dimensions $R^{d}$ as the original word representations.

\textcite{edilson_a._correa_nilc-usp_2017} expanded upon this simple yet effective idea to tackle the SemEval-2017 Task 4\footnote{\url{http://alt.qcri.org/semeval2017/task4}}, Sentiment Analysis in Twitter.
In order to acquire embeddings that represented \emph{tweets}, they weighed the word embeddings that made up a tweet; $\text{tweet}_i = (w_{i1}, w_{i2}, \dots, w_{im})$ with the \tfidf{} weights.
For the \tfidf{} calculation, they cast individual weights as documents so that term frequency become the term count in a single tweet while document frequency become the number of tweets the term $w_t$ occurs.

We have mentioned that our dictionary definitions are not continuous.
Yet, we advocate using \tfidf{} weights to weigh our word embeddings to get sentence embeddings.
In order to clarify, let us present Table~\ref{tab:en_it_examples}.
\begin{table}
    \centering
    \caption{Some definitions from English Princeton WordNet}%
    \label{tab:en_it_examples}
    \begin{tabular}{l}
        \toprule
        turn red, as if in embarrassment or shame \\
        a feeling of extreme joy \\
        a person who charms others (usually by personal attractiveness) \\
        so as to appear worn and threadbare or dilapidated \\
        a large indefinite number \\
        distributed in portions (often equal) on the basis of a plan or purpose \\
        a lengthy rebuke \\
        \bottomrule
    \end{tabular}
\end{table}

% For sentence embeddings, first a \tfidf{} matrix is constructed.
For the \tfidf{} calculations, we followed a similar approach.
The term frequency is the raw count of a term in a dictionary definition.
While the document frequency is the number of dictionary definitions where $w_t$ occurs.

Then, with the term-embedding matrix at hand, we have calculated definition embeddings using;
\begin{equation}
    S_{\text{emb}}(S) = \sum_{w_{i} \in S} \varB{tf_{w_{i},S}-idf_{w_i}} \cdot Emb_{w}(w_{i})
\end{equation}
Every word that makes up a definition is scaled by its vector in ${\rm I\!R}^n$, then concatenated to form sentence embeddings on ${\rm I\!R}^n$.
% TODO cost matrix
% Bipartite graph or matrix
% Give the assumption, defintions are one-to-one
Given the N vectors from source and target language, we hypothesize that there exists a matching where every source definition vector is perfectly mapped to one target vector.
Given that this problem naively iterates over $N!$ matchings, we have looked into an algorithm.

%%% TODO lapjv %%%
% https://blog.sourced.tech/post/lapjv/
%
\section{Linear Assignment Algorithm}%
\label{sec:linear_assignment_algorithm}
% Ok jonker volgenant is super complicated
% can I just say I'm using linear assignment? maybe talk about hungarian algortihm a bit



\section{Results}%
\label{sec:results}

\begin{table}[htbp]
    \centering
    \begin{tabular}{lrrr}
        \toprule
& \multicolumn{3}{c}{Percentage of Correctly Matched Definitions} \\
\cmidrule(lr){2-4}
        \textbf{Language} & \textbf{fastText 1M} & \textbf{fastText 500k} & \textbf{Numberbatch} \\
        \midrule
        bg & 0.39 & 0.41 & 0.19 \\
        el & 0.37 & 0.38 & 0.14 \\
        it & 0.28 & 0.28 & 0.36 \\
        ro & 0.39 & 0.39 & 0.20 \\
        sl & 0.15 & 0.15 & 0.06 \\
        sq & 0.55 & 0.54 & 0.27 \\
        \bottomrule
    \end{tabular}
    \caption{Linear Assignment Using 2000 Definitions}%
    \label{tab:lapjv_2000}
\end{table}

\begin{table}[htbp]
    \centering
    \begin{tabular}{lrrr}
        \toprule
& \multicolumn{3}{c}{Percentage of Correctly Matched Definitions} \\
\cmidrule(lr){2-4}
        \textbf{Language} & \textbf{fastText 1M} & \textbf{fastText 500k} & \textbf{Numberbatch} \\
        bg & 0.35 & 0.36 & 0.18 \\
        el & 0.36 & 0.36 & 0.12 \\
        it & 0.25 & 0.25 & 0.32 \\
        ro & 0.36 & 0.37 & 0.19 \\
        sl & 0.11 & 0.11 & 0.05 \\
        sq & 0.39 & 0.40 & 0.19 \\
        \bottomrule
    \end{tabular}
    \caption{Linear Assignment Using 3000 Definitions}%
    \label{tab:lapjv_3000}
\end{table}

\begin{table}[htbp]
    \centering
    \begin{tabular}{lrrr}
        \toprule
& \textbf{fastText 1M} & \textbf{fastText 500k} & \textbf{Numberbatch} \\
\midrule
        Best & 0.55 & 0.54 & 0.36 \\
        Worst & 0.11 & 0.11 & 0.05 \\
        Average & 0.33 & 0.33 & 0.19 \\
        \bottomrule
    \end{tabular}
    \caption{Summary of Linear Assignment}%
    \label{tab:lapjv_summary}
\end{table}
