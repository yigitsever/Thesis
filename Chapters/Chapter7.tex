% Chapter Template

\chapter{Conclusion}%
\label{chap:conclusion}

In this study, we set out to investigate the feasibility of representing senses using their dictionary definitions.
Along the way, we used document retrieval, linear programming and neural networks to answer the issue on as many angles as possible.
The grand aim of the study was to compare the approaches that we had identified for the task.
To our best knowledge, a comparable study where the dictionary alignment approaches were reported on the basis of their performance is not available so we had to anchor the study to itself.
At the end of the day, we can make justified comparisons.

The monolingual retrieval using \tfidf weights and cosine similarity measure was chosen as a baseline because it is the most greedy approach available.
If dictionary generation could be solved by automatic machine translation, this thesis would not take hold.
The results presented in Chapter~\ref{chap:experiments_and_evaluation} prove so.

The matching algorithm is interesting.
Moving on with our greedy connotation, for a task like dictionary alignment, assigning a sense to a definition that is closest to it by some distance metric might leave another definition with less than an ideal match later down the line.
Matching ensures that the \emph{closest} metric in between definitions holds not just for individual definitions but for the whole corpora.
We can refer to Figure~\ref{fig:bipartite_graph} to illustrate this point.

We have mentioned the lexical gap problem in Chapter~\ref{chap:background_n_related} where some senses do not have equivalences in the target language.
Recently, \textcite{bolukbasi_man_2016} reported on gender biases of word embedding models which numberbatch embeddings responded with so called de-biased embeddings, eliminating it from their models almost completely.
Considering the most common type of lexical gap arises from languages with grammatical gender, possible effect of this on the matching approach is left for future work.

Overall, matching approach consistently shown the best performance across the board, supporting our hypothesis that one-to-one matching two sets of dictionary definitions would result in superior performance.
We have also proven our justification behind the choice of the particular embedding model and the fact that conventional evaluation of word embeddings might not translate to downstream tasks.
Numberbatch has scored first place on SemEval-2017 Task 2~\cite{camachocollados-EtAl:2017:SemEval}, on multilingual word similarity task.
Yet, against fastText embeddings, their model performed worse with the exception of Italian.
Italian is a \emph{core language} for numberbatch, where they claim full support.
It is also the language where numberbatch consistently outperformed fastText embeddings.
We have set out to investigate the effect of particular choices like this for the dictionary alignment task.
It can be reported with confidence that the advantage of one embedding model over another is not clear cut and should be investigated further.

With the supervised long short-term memory approach, we have observed that not only it is possible to represent senses using their dictionary definitions but also the metric of \emph{representing} the same sense can be learned.
The data required for obtaining any good performance should be noted and experimenting on diverse data should be left for future work.

The crucial shortcoming is the data requirements we have.
On one hand, any type of description that represent a sense can be aligned not with just WordNet but any dictionary.
Projects like BabelNet\footnote{\url{https://babelnet.org}} or ConceptNet are creating semantic databases of their own while WordNet is on version 3.0, still online well after 20 years.
Natural language processing research relies on external sources of information and the pre-annotated nature of these resources will always find a use.
Working towards automatically extending them creates more opportunities for sprawling research later down the line.

Our main contribution in this study is the empirical comparison of alignment and retrieval approaches.
We have hypothesized that aligning definitions one-to-one instead of greedily assigning each definition to it's closest counterpart will perform better.
Our intuition behind the hypothesis is that dictionaries include discrete senses.
Once a pair of definitions is matched, continuing to align further senses to any of the definitions can only deteriorate the performance.
The results we have presented in Section~\ref{sec:case_study} confirms our hypothesis.
Matching approaches outperformed retrieval approaches on any language set.
Including 6 different languages and observing the performance differences on all of them further confirms that by using the power of word embeddings, our finding are as language agnostic as possible.
Our final conclusion is that the state of the art approach Sinkhorn distance~\cite{balikasCrosslingual2018} between term document representation outperformed sentence embeddings that were proposed specifically for short text representation.
Further studies in the field can take this finding into account in their models.

\section{Future Work}%
\label{sec:future_work}

Throughout the thesis, English was always the centrepiece of the experiments.
The wordnets were evaluated by their alignment towards the first and the most comprehensive, WordNet.
The word embeddings were mapped to share a latent space with English word embeddings.
As we have mentioned in Chapter~\ref{chap:background_n_related}, ideas like Inter-Lingual Index offer ways to bypass the English as a hub language.
As an immediate future work, alignments that do not use English nor English Princeton WordNet can be investigated.
Culturally or syntactically closer languages can be bridges more easily than distant yet abundant English.

Recent transfer learning models like BERT~\cite{devlin_bert_2018} offer a novel way to overcome the fundamental shortcoming with the supervised encoder we presented;
the model performs in accordance with the available data and requires aligned data to function in the first place.
Transfer learning inspires approaches like encoding the metric for representing the same sense in $n$ languages after which the model is ready to predict on $n+1^{th}$ language.
Very recently, \textcite{jawanpuriaLearning2019} proposed a VecMap like framework for convenient alignment of word embeddings.
To our interest, the framework can map \emph{multilingual} embeddings on a \emph{shared space}.
With a potential synset discovery approach like the one proposed by \textcite{ruiz-casado_automatic_2005} where possible sense definitions are found and validated using supervised learning will be investigated next using the novel ideas as inspiration.
