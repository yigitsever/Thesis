%!TEX TS-program = xelatex
%%%%%%%%%%%%%%%%%%%%%%%%%%%%%%%%%%%%%%%%%
% Masters/Doctoral Thesis
% LaTeX Template
% Version 2.5 (27/8/17)
%
% This template was downloaded from:
% http://www.LaTeXTemplates.com
%
% Version 2.x major modifications by:
% Vel (vel@latextemplates.com)
%
% This template is based on a template by:
% Steve Gunn (http://users.ecs.soton.ac.uk/srg/softwaretools/document/templates/)
% Sunil Patel (http://www.sunilpatel.co.uk/thesis-template/)
%
% Template license:
% CC BY-NC-SA 3.0 (http://creativecommons.org/licenses/by-nc-sa/3.0/)
%
%%%%%%%%%%%%%%%%%%%%%%%%%%%%%%%%%%%%%%%%%

%----------------------------------------------------------------------------------------
%	PACKAGES AND OTHER DOCUMENT CONFIGURATIONS
%----------------------------------------------------------------------------------------

\documentclass[
12pt, % The default document font size, options: 10pt, 11pt, 12pt
%oneside, % Two side (alternating margins) for binding by default, uncomment to switch to one side
english, % ngerman for German
onehalfspacing, % Single line spacing, alternatives: {onehalfspacing} or doublespacing -> one half spacing is mandated
%draft, % Uncomment to enable draft mode (no pictures, no links, overfull hboxes indicated)
nolistspacing, % If the document is onehalfspacing or doublespacing, uncomment this to set spacing in lists to single
%liststotoc, % Uncomment to add the list of figures/tables/etc to the table of contents
%toctotoc, % Uncomment to add the main table of contents to the table of contents
parskip, % Uncomment to add space between paragraphs
%nohyperref, % Uncomment to not load the hyperref package
headsepline, % Uncomment to get a line under the header
%chapterinoneline, % Uncomment to place the chapter title next to the number on one line
consistentlayout, % Uncomment to change the layout of the declaration, abstract and acknowledgements pages to match the default layout
]{MastersDoctoralThesis} % The class file specifying the document structure
\usepackage{fontspec}
\usepackage{polyglossia}
\setdefaultlanguage{english}
\setmainfont{Lora}
\usepackage[bottom]{footmisc} % prevent figures from appearing under footnotes
\usepackage[backend=biber,style=nature,natbib=true]{biblatex} % Use the bibtex backend with the authoryear citation style (which resembles APA)
\addbibresource{thesis.bib} % The filename of the bibliography

\usepackage[autostyle=true]{csquotes} % Required to generate language-dependent quotes in the bibliography

% correct spacing in tf-idf
\newcommand{\varA}[1]{{\operatorname{#1}}}
\newcommand{\varB}[1]{{\operatorname{\mathit{#1}}}}
\newcommand{\tfidf}{\ $\varB{tf-idf}$\ }

\usepackage[super]{nth} % 1st 2nd 3rd 4th ...
\usepackage{booktabs}   % beautiful tables
\usepackage{tabulary}
\usepackage{multirow}
% \usepackage{amsmath}    % general purpose math, mathtools load it
\usepackage{amsfonts,amssymb}
\usepackage{mathtools}
\DeclarePairedDelimiter{\ceil}{\lceil}{\rceil} % ceiling function wrapper

\usepackage{url}        % add urls to footnotes
\emergencystretch=1em
\RequirePackage[xetex]{hyperref}
\usepackage{todonotes}

\newcommand{\titletitlesize}{\fontsize{16pt}{18pt}\selectfont}
%----------------------------------------------------------------------------------------
%	MARGIN SETTINGS
%----------------------------------------------------------------------------------------

\geometry{
	paper=a4paper, % Change to letterpaper for US letter
	inner=2.5cm, % Inner margin
	outer=2.5cm, % Outer margin
	bindingoffset=.5cm, % Binding offset
	top=2.5cm, % Top margin
	bottom=2.5cm, % Bottom margin
	%showframe, % Uncomment to show how the type block is set on the page
}   % hacettepe recommended values

%----------------------------------------------------------------------------------------
%	THESIS INFORMATION
%----------------------------------------------------------------------------------------

\thesistitle{Evaluating Bilingual Embeddings in Bilingual Dictionary Alignment} % Your thesis title, this is used in the title and abstract, print it elsewhere with \ttitle
%\turkishtitle{Çiftdilli Kelime Temsilleri ile Sözlük Eşlenmesi} % Your thesis title, this is used in the title and abstract, print it elsewhere with \ttitle
\supervisor{Asst.~Prof.~Dr.~Gönenç \textsc{Ercan}} % Your supervisor's name, this is used in the title page, print it elsewhere with \supname
\examiner{} % Your examiner's name, this is not currently used anywhere in the template, print it elsewhere with \examname
\degree{Master of Science} % Your degree name, this is used in the title page and abstract, print it elsewhere with \degreename
\author{Yiğit \textsc{Sever}} % Your name, this is used in the title page and abstract, print it elsewhere with \authorname
\addresses{} % Your address, this is not currently used anywhere in the template, print it elsewhere with \addressname
\subject{Computer Engineering} % Your subject area, this is not currently used anywhere in the template, print it elsewhere with \subjectname
\keywords{} % Keywords for your thesis, this is not currently used anywhere in the template, print it elsewhere with \keywordnames
\university{\href{http://www.university.com}{Hacettepe University}} % Your university's name and URL, this is used in the title page and abstract, print it elsewhere with \univname
\department{\href{http://department.university.com}{Department of Computer Engineering}} % Your department's name and URL, this is used in the title page and abstract, print it elsewhere with \deptname
%\group{\href{http://researchgroup.university.com}{Research Group Name}} % Your research group's name and URL, this is used in the title page, print it elsewhere with \groupname
\faculty{\href{http://faculty.university.com}{Graduate School of Science and Engineering}} % Your faculty's name and URL, this is used in the title page and abstract, print it elsewhere with \facname

\makeatletter
\hypersetup{%
    final       = true,
    colorlinks  = true,
    urlcolor    = blue,
    citecolor   = blue,
    linkcolor   = MidnightBlue,
    unicode     = true,
    linktoc     = section,
    pdflang     = en-GB,
    pdfauthor   = {\authorname},
    pdfkeywords = {\keywordnames},
    pdftitle    = {\ttitle},
    pdfsubject  = {}
}
\makeatother

% \AtBeginDocument{
% \hypersetup{pdftitle=\ttitle} % Set the PDF's title to your title
% \hypersetup{pdfauthor=\authorname} % Set the PDF's author to your name
% \hypersetup{pdfkeywords=\keywordnames} % Set the PDF's keywords to your keywords
% }

\begin{document}

\frontmatter % Use roman page numbering style (i, ii, iii, iv...) for the pre-content pages

\pagestyle{plain} % Default to the plain heading style until the thesis style is called for the body content

%----------------------------------------------------------------------------------------
%	TITLE PAGE
%----------------------------------------------------------------------------------------

\begin{titlepage}
\begin{center}
\vspace*{2\baselineskip} % required by template
{\bfseries \titletitlesize{\ttitle}\par} % Thesis title
\vspace*{3\baselineskip} % required by template
{\bfseries \titletitlesize{Çift Dilli Kelime Temsilleri ile Sözlük Eşlenmesi}\par} % Thesis title
\vspace*{3\baselineskip} % required by template
{\bfseries Yiğit Sever\par}
\vspace*{3\baselineskip} % required by template
{\bfseries Dr.~Gönenç Ercan\par} % Supervisor name - remove the \href bracket to remove the link
{\bfseries Supervisor}

\vspace*{3\baselineskip} % required by template

Submitted to\\
Graduate School of Science and Engineering of Hacettepe University\\
as a Partial Fulfilment to the Requirements\\
for the Award of the Degree of Master of Science\\
in Computer Engineering

\vspace*{3\baselineskip} % required by template

2019

\end{center}
\end{titlepage}

%----------------------------------------------------------------------------------------
%	DECLARATION PAGE
%----------------------------------------------------------------------------------------

\begin{declaration}
\addchaptertocentry{\authorshipname} % Add the declaration to the table of contents
\noindent I, Yiğit Sever, declare that this thesis titled, \enquote{\ttitle} and the work presented in it are my own. I confirm that:

\begin{itemize}
\item This work was done wholly or mainly while in candidature for a research degree at this University.
\item Where any part of this thesis has previously been submitted for a degree or any other qualification at this University or any other institution, this has been clearly stated.
\item Where I have consulted the published work of others, this is always clearly attributed.
\item Where I have quoted from the work of others, the source is always given. With the exception of such quotations, this thesis is entirely my own work.
\item I have acknowledged all main sources of help.
\item Where the thesis is based on work done by myself jointly with others, I have made clear exactly what was done by others and what I have contributed myself.
\end{itemize}

\noindent Signed:\\
\rule[0.5em]{25em}{0.5pt} % This prints a line for the signature

\noindent Date:\\
\rule[0.5em]{25em}{0.5pt} % This prints a line to write the date
\end{declaration}

%\cleardoublepage{}

%----------------------------------------------------------------------------------------
%	QUOTATION PAGE
%----------------------------------------------------------------------------------------

% buraya muhtemelen gerek yok
%\vspace*{0.2\textheight}

%\noindent\enquote{\itshape{} Thanks to my solid academic training, today I can write hundreds of words on virtually any topic without possessing a shred of information, which is how I got a good job in journalism.}\bigbreak{}

%\hfill Dave Barry

%----------------------------------------------------------------------------------------
%	ABSTRACT PAGE
%----------------------------------------------------------------------------------------

% Abstract icime cok guzel sindi hocam burayi son draft olarak dusunebilirsiniz
\begin{abstract}
\addchaptertocentry{\abstractname} % Add the abstract to the table of contents
\noindent{}Dictionaries catalog and describe the semantic information of a lexicon.
WordNet provides an edge by presenting distinct concepts with the hierarchy information among them.
Research in computer science has been using this hand crafted tool in natural language applications such as text summarization and machine translation.
Original WordNet has been compiled for English yet counterparts for other languages are not as readily available nor as comprehensive.
In order for research on languages other than English to benefit from the power of a WordNet, machine assisted creation and evaluation methods are essential.

Word embeddings can provide a mapping between words and points in a real valued vector space.
Using these vectors, representing documents as well as forming geometric relationships between them is a well studied area of research.
In this thesis we start by hypothesizing that a dictionary definition captures the semantic basis of the described word.
We used word embeddings as building blocks to map dictionary definitions into a multidimensional space.
These spaces can be aligned to accommodate two languages, allowing the transfer of information from one language to another.
We investigate the success of retrieving and matching discrete senses across languages by employing supervised and unsupervised methods.
Our experiments show that dictionary alignment can be evaluated successfully by using both unsupervised and supervised methods but corpora sizes should be taken into consideration.
We further argue that some methods are not viable considering their poor performance.

\end{abstract}

%----------------------------------------------------------------------------------------
%	ACKNOWLEDGEMENTS
%----------------------------------------------------------------------------------------

\begin{acknowledgements}
\addchaptertocentry{\acknowledgementname} % Add the acknowledgements to the table of contents
The acknowledgments and the people to thank go here, don't forget to include your project advisor\ldots
\end{acknowledgements}

%----------------------------------------------------------------------------------------
%	LIST OF CONTENTS/FIGURES/TABLES PAGES
%----------------------------------------------------------------------------------------

\tableofcontents % Prints the main table of contents

\listoffigures % Prints the list of figures

\listoftables % Prints the list of tables

%----------------------------------------------------------------------------------------
%	ABBREVIATIONS
%----------------------------------------------------------------------------------------

% \begin{abbreviations}{ll} % Include a list of abbreviations (a table of two columns)

% \textbf{LAH} & \textbf{L}ist \textbf{A}bbreviations \textbf{H}ere\\
% \textbf{WSF} & \textbf{W}hat (it) \textbf{S}tands \textbf{F}or\\

% \end{abbreviations}

%----------------------------------------------------------------------------------------
%	PHYSICAL CONSTANTS/OTHER DEFINITIONS
%----------------------------------------------------------------------------------------

%\begin{constants}{lr@{${}={}$}l} % The list of physical constants is a three column table

%% The \SI{}{} command is provided by the siunitx package, see its documentation for instructions on how to use it

%Speed of Light & $c_{0}$ & \SI{2.99792458e8}{\meter\per\second} (exact)\\
%%Constant Name & $Symbol$ & $Constant Value$ with units\\

%\end{constants}

%----------------------------------------------------------------------------------------
%	SYMBOLS
%----------------------------------------------------------------------------------------

%\begin{symbols}{lll} % Include a list of Symbols (a three column table)

%$a$ & distance & \si{\meter} \\
%$P$ & power & \si{\watt} (\si{\joule\per\second}) \\
%%Symbol & Name & Unit \\

%\addlinespace % Gap to separate the Roman symbols from the Greek

%$\omega$ & angular frequency & \si{\radian} \\

%\end{symbols}

%----------------------------------------------------------------------------------------
%	DEDICATION
%----------------------------------------------------------------------------------------

\dedicatory{For/Dedicated to/To my\ldots}

%----------------------------------------------------------------------------------------
%	THESIS CONTENT - CHAPTERS
%----------------------------------------------------------------------------------------

\mainmatter{} % Begin numeric (1,2,3...) page numbering

\pagestyle{thesis} % Return the page headers back to the "thesis" style

% Include the chapters of the thesis as separate files from the Chapters folder
% Uncomment the lines as you write the chapters

\chapter{Introduction}\label{chap:introduction}%
% Problem tanimi
% Neden cozmek istiyoruz
% Dictionaries take time and effort to make you can cite life of Webster or ask about Turkish lexiconners?
% Dictionaries are important
% talk about dictionaries
\section{Dictionaries}%
\label{sec:dictionaries}
Dictionary is a broad concept to define since lexicographers prepare them for various specific use cases.
For instance, bilingual dictionaries present words alongside their translations in the target language.
Domain specific dictionaries list technical terms that target people who are familiar with the terminology.
Yet, the term \emph{dictionary} on its own brings forth the monolingual type into consideration.
This type of dictionary presents words alongside their definitions following an alphabetical order~\cite{sterkenburg_practical_2003}.
Their primary intention is to inform the user about the words~\cite{uzun_modern_2005}.
Some words are polysemous, sharing the same spelling and having related, often derivative meanings.
These different senses are clarified through multiple definition entries.
On the other hand, homonymous words have distinct meanings while having identical spellings through coincidence.
They are often denoted using discrete blocks of descriptions.
Finally, the last lexical relation is the synonymity.
A word is synonymous to another if they share the same meaning but are not spelled alike.
The term that precede the entries are called \emph{headword} or \emph{lemma}.
Usually, lemmas are the form of a word without inflections or derivations.
The sense they convey is as comprehensive as possible, reducing the number of otherwise redundant entries that would be the derivatives of the unmarked form~\cite{ibrahim_usta_turkce_2006}.

% from dictionaries to multilingual web
Dictionaries take an immense amount of time and expertise to prepare.
We can list the examples after narrowing our scope down to the dictionaries that people still use today.
A survey by \textcite{uzun_1945ten_1999} notes that the first instalment of the modern Turkish dictionary, led by a team of experts, has taken over 6 years to prepare.
\textcite{kendall_forgotten_2011} talks about how Noah Webster, the writer of the \emph{An American Dictionary of the English Language} had to mortgage off his home in order to finish his project which took over 26 years.
Early lexicographers had to work for years in order to collect material to prepare a corpus~\cite{uzun_1945ten_1999}.
This endeavour was necessary since a corpus is crucial to create the vocabulary of a language, otherwise known as a \emph{lexicon}.

The internet radically changed the way researchers aggreate the data.
The advancements in digital storage technology allowed the data to be persistent.
Improvements in networking ensured that people can share the volume of it between themselves.
With the popularization of the social media, the internet generates everyday conversations that are useful in natural language applications.
On the other hand, efforts on open, collaborative, web based encyclopedias generate structured, multilingual data.
Once cumbersome task of corpus attainment is now akin to web crawling.

With the digitized data, it was only natural for dictionaries to go digital as well.
% start wordnets
\section{WordNet}% not sold on having wordnet as a section
\label{sec:wordnet}
WordNet~\cite{fellbaum_wordnet_1998-1} is a lexical database.
In the context of dictionaries, we have emphasized lemmas and definitions.
WordNet can show relationships like hyponymym hierarchy transitive or meronymym part-whole relation, only possible with an electronic resource.

\enquote{WordNet is a semantic network}\cite{fellbaum_wordnet_1998-1}

Authorities list more than 7000 living languages\footnote{\url{https://www.ethnologue.com/statistics}} with only 40\footnote{\url{https://w3techs.com/technologies/history_overview/content_language/}} of them having a sizeable presence on the internet.
Nevertheless, among this small fraction, English is the dominant language of the web.

Translation, information transfer from foreign languages is a valid way of enriching a language's corpora~\cite{ibrahim_usta_turkce_2006}.

We should note that the languages of the wordnets used in the thesis are all present in the 40 languages that have a significant presence on the internet that we have mentioned before.
Linking wordnets leads to more research~\cite{sagot_building_2008}.

\textcite{fellbaum_semantic_1998} defines the correct terminology that we abide for the thesis; \enquote{As WordNet became synonymous with a particular kind of lexicon design, the proper name shed its capital letters and became a common designator for semantic networks of natural languages}.
Hence \emph{WordNet} refers to English Princeton WordNet, while wordnets created for other languages are not stylized.

Further research in the area contributes to more languages having access to tools that will incorporate them into the literature.
English in not the centrepiece for most natural language processing research because it is the language that can store the most information or any other linguistic advantage.
It's the most abundant language on web.
Distributions like spaCy resorts to lemmatizations like =PRON= to denote pronouns in order to collapse the senses for "I", "you", "them" etc.\@.
% might be too cheesy
The sense and the accompanying word for being the brother of someone's father or mother differs in Turkish.\footnote{\emph{Amca} for brother of the father and \emph{Dayı} for brother of the father.}
Yet no distinction exists for English and senses collapse together in \emph{uncle}.
% \might be too cheesy


James Somers puts down the modern dictionaries by saying \enquote{The definitions are these desiccated little husks of technocratic meaningese, as if a word were no more than its coordinates in semantic space.}\cite{somers_youre_2014}.

The work on wordnets for languages other than English has been under way since the early days of the predecessor.

Research has tackled the issue of lack of wordnets for languages besides English. %TODO reference

Yet, against %TODO how many?
senses in the English Princeton WordNet, the most comprehensive counterpart remain at ... % TODO how many
Coupled with the licensing issue which prevents scientific research from using some WordNets.
We have constrained this study to use only the freely available wordnets.
Open Multilingual WordNet~\cite{bond_survey_2012} presents wordnets from other languages with three crucial additions; % TODO the following sentence is plagiarism
They have normalized the data, aligned with English Princeton WordNet and they are all accessible from a single source.\footnote{\url{http://compling.hss.ntu.edu.sg/omw/}}
With alignment information at hand, we have a way to create a golden corpora that we assume to be perfectly aligned.
Among the 34 wordnets available on Open Multilingual WordNet, only 6 of them have gloss information available.
Given this thesis will only investigate the ability to map senses using definitions of the sense, we used the subset of Albanian~\cite{ruci_current_2008}, Bulgarian~\cite{simov_constructing_2010}, Greek~\cite{stamou_exploring_2004}, Italian~\cite{pianta_multiwordnet_2002}, Slovenian~\cite{fiser_slownet_2012} and Romanian~\cite{tufis_romanian_2008}.
\begin{table*}[!hbp]
    \begin{center}
        \caption{Summary of the Wordnets used.}\label{tab:summary_table}
        \begin{tabular}{llrr}
            \toprule%
            \textbf{Name of the Project} & \textbf{Language} & \textbf{Number of Definitions} & \textbf{Words Per Definition} \\
            \midrule%
            Albanet & Albanian & 4681 & 11.75 \\
            BulTreeBank WordNet & Bulgarian & 4959 & 12.71 \\
            % DanNet & Danish    & 784                   & 8.63                           \\
            Greek Wordnet & Greek & 18136 & 11.24 \\
            ItalWordnet & Italian & 12688 & 7.33 \\
            Romanian Wordnet & Romanian & 58754 & 9.98 \\
            SloWNet & Slovenian & 3144 & 12.68 \\
            \bottomrule %
        \end{tabular}
    \end{center}
\end{table*}


% Chapter Template

\chapter{Background Information \& Related Work}\label{chap:background_n_related}
% ROUGH DRAFT

James Somers puts down the modern dictionaries by saying \enquote{The definitions are these desiccated little husks of technocratic meaningese, as if a word were no more than its coordinates in semantic space.}~\cite{somers_youre_2014}.
Even though the author criticises the efficient of the dictionary definitions, we will build the thesis on the idea that we can represent senses using their dictionary definitions.

\section{Word Embeddings}%
\label{sec:word_embeddings}

Recent studies have been using word representations, commonly known as \emph{word embeddings}.
Word embeddings are real valued, dense feature vectors for words.
They are induced in order to map a lexicon to a multidimensional latent space.
This representation allows researchers access to the tools of a broad literature in linear algebra and machine learning.
Since the embeddings and their respective words (labels) can be saved to the disk, researchers have been sharing their models on the internet for other researchers to simply download and use them on their own applications.
Word embeddings acquired this way are often called \emph{pre-trained}.

In this section, we will present a brief history of word embeddings.
At the end of the section, we will study our selected model, \emph{fasttext}~\cite{mikolov2018advances}.

Word embeddings is a sprawling subject that has been built upon ideas from probabilistic, statistical and neural network models.
We have omitted approaches that are not used for our study and constrained ourselves only to the literature that lead up to the model we will use. % dilimiz dondugunce

\subsection{History of Word Representations}%
\label{sub:history_of_word_representations}

In order to talk about how words can be mapped to a multidimensional space, first we should talk about how the idea that they can has been theorized.

\subsubsection{Linguistic Background}%
\label{ssub:linguistic_background}

In his \citeyear{harris_distributional_1954} article, \textcite{harris_distributional_1954} introduced his ideas which later came to known as \emph{distributional hypothesis} in the field of linguistics.
He argued that similar words appear within similar contexts.
The famous quote by \textcite{firth_synopsis_1957} captures the idea as;
\enquote{You shall know a word by the company it keeps!}
For instance, the semantic similarity between the terms \emph{jacket} and \emph{coat} can be theoretically proven since they will be accompanied by similar verbs, such as \emph{wear}, \emph{dry clean} or \emph{hang}, and similar adjectives such as \emph{warm} or \emph{leather}.
% is leather an adjective?
However, for a researcher to extract these rules by hand would have been infeasible.

Even though \citeauthor{harris_distributional_1954} argued that \enquote{language is not merely a bag of words}, using unordered collection of word counts to capture the semantic information will be used in the literature and be known as the \emph{bag-of-words} hypothesis.

\subsubsection{Vector Space Model}%
\label{ssub:vector_space_model}

The history of word embeddings is tightly coupled with vector space models.
The vector space models first appeared in the information retrieval field.
Initial vector space model developed by \textcite{salton_vector_1975} and presented in \citetitle{salton_vector_1975}.
It was the first application of bag-of-words hypothesis on a corpus to extract semantic information~\cite{turney_frequency_2010}.
\citeauthor{salton_vector_1975} presented the novel idea of a \emph{document space}, consisting of fixed sized vectors as the columns of a term document matrix.
The dimensions of the vectors were the whole vocabulary of the corpus.

In this space, a document $D_i$ is represented using $t$ distinct terms as a row vector;
\begin{displaymath}
    D_{i} = (d_{i1}, d_{i2}, \ldots, d_{it})
\end{displaymath}
The weights for the index terms are calculated by using the \tfidf{} measure introduced by \textcite{jones_statistical_1972}.
\tfidf{} is the multiplication of two metrics;
\begin{description}
    \item[\emph{tf}] the number of times a term $k$ occurs in a document
    \item[\emph{idf}] the inverse of the number of documents that contain $k$.
\end{description}

\citeauthor{salton_vector_1975} presented their particular weighting scheme where the term frequency is multiplied by the following inverse document frequency for the term $k$.
\begin{displaymath}
IDF_{k} = \ceil[\big]{\log_{2}n} - \ceil[\big]{\log_{2}d_{k}} + 1
\end{displaymath}
% simply log_2(n/d_{k})
Where $n$ is the number of documents in the collection and $d_k$ is the number of documents that consists the term $k$.
The weighting scheme was selected to \enquote{assign the largest weight to those terms which arise with high frequency in individual documents, but are at the same time relatively rare in the collection as a whole}.
Finally, they have cast their similarity function between documents $i$ and $j$, as the inner product between their vectors which corresponds to the cosine similarity.

The vector space model allowed \citeauthor{salton_vector_1975} to handle the similarity between documents as the angle between two vectors.
More importantly, they have shown that there is merit to handling documents as real valued vectors.

\subsubsection{Latent Semantic Analysis}%
\label{ssub:latent_semantic_analysis}

\textcite{deerwester_indexing_1990} introduced latent semantic analysis in order to address a crucial problem with the vector space model.
They have identified that synonyms and homonyms cannot be handled by the naive term document matrix approach due to the fact that vector space model requires the words to match exactly between the two documents.
Synonymity is an issue because the query might have terms that have the same meaning as the target word.
On the other hand, homonyms might match with an unrelated word.
Their model seeks the higher order latent semantic structure in order to learn the \emph{similarity between words}.

Latent semantic analysis starts with a word co-occurrence matrix $X$.
The terms of the matrix is weighted by an arbitrary weighting scheme, such as \tfidf{}, pointwise mutual information\cite{church_word_1990} or raw term frequencies as used in the original study.
% Ruder Survey claims pmi weights, original paper uses raw frequencies, says it'll be chosen per application anyway
A term document matrix $X$ is then factorized into three matrices using singular value decomposition~\cite{forsythe_computer_1977};
% People don't cite this for some reason, is it wrong?
\begin{displaymath}
    X = T_{0}S_{0}D_{0}'
\end{displaymath}
Where the columns of $T_{0}$ and $D_{0}'$ are orthogonal to each other and $S_{0}$ is the diagonal matrix of singular values.
The singular values of $S_{0}$ can be ordered by size to keep only the $k$ largest elements, setting the others to zero~\cite{deerwester_indexing_1990}.

Their use for the resulting matrix is as follows;
\begin{displayquote}
Each term or document is then characterized by a vector of weights indicating its strength of association with each of these underlying concepts.
That is, the \enquote{meaning} of a particular term, query, or document can be expressed by $k$ factor values, or equivalently, by the location of a vector in the $k$-space defined by the factors.
\end{displayquote}
They have used this technique to solve document similarity task and touched upon \emph{word similarity}.
% TODO I can't talk about this paper in a concise way or any way really, return later
Using latent semantic analysis to represent word similarity is later studied by \textcite{landauer_solution_1997}

\subsubsection{Building Upon Distributional Hypothesis}%
\label{ssub:building_upon_distributional_hypothesis}

% \textcite{schutze_dimensions_1992} (intro to information retrieval co-author) is earlier but doesn't appear in any survey, A Survey on Word Representations of Meaning cites later works (Automatic word sense disambiguation), most just skip to lund
% "in order for the dimensions of meaning and the vector representations of words to reflect closeness in meaning faithfully, a global optimization of cooccurrence constraints is necessary, an operation so complex that only a supercomputer can perform it." that's why
% also used a context size of _characters_ because "few long words are better than many short words"
While \citeauthor{deerwester_indexing_1990} studied relatedness between words using vectors, their approach used the whole corpus for the co-occurrence information and the focus was still on the document similarity.

\textcite{schutze_dimensions_1992} proposed \enquote{to represent the semantics of words and contexts in a text as vectors} and built upon word co-occurrence.
They theorized a context window of 1000 characters in order to not consider the whole corpus but only words that are close to the target word to be considered in co-occurrence calculations.
% They used cosine similarity as a measure of semantic relatedness between word vectors.
However, they claimed that the computation power available was not suitable yet to fully tackle the task.

% Lund is not cited as much, is context window important?
\textcite{lund_producing_1996} took the challenge and experimented upon 160 million words taken from the internet.
They used a context size of 10 words and provided a method to obtain feature vectors to represent the meaning of words.
% maybe talk about the method but seems redundant
However, intricate tuning of word co-occurrence generated associatively similar vectors instead of semantically similar ones.
% Figure 1 of this paper can be adapted here

\subsubsection{Neural Network Models}%
\label{ssub:neural_network_models}

% \textcite{bengio_neural_2000} root word embedding paper.
% There is another paper called A Neural Probabilistic Language Model, that came out in 2000
% People pretend that it doesn't exist
\textcite{bengio_neural_2000} proposed the first neural network model.
Neural network models will be the centrepiece in word embedding research later.
They wanted to pursue the curse of dimensionality initially because the corpora were getting bigger and term document matrices were \ldots.
They claimed that "we use a continuous real-vector..."

The idea presented by \citeauthor{lund_producing_1996} similar words should have similar feature vectors is presented here.
\citeauthor{lund_producing_1996} has shown this hypothesis with co-occurrence vectors and \citeauthor{bengio_neural_2000} used a distributed feature vector, learned by a probability distribution.

\citeauthor{collobert_unified_2008} suggested a deep neural network model to solve various natural language processing tasks but relevant to our study, have proposed to explicitly learn the feature vectors at the same time.
% something about unsup. learning

They have also used a window that looked ahead and behind of the target word instead of previous methods which have traditionally only looked up to the word, sticking to $P(w_t | w_{t+1})$.
Jointly with \citeauthor{collobert_unified_2008}, \citeauthor{p._turian_word_2010} steered the work on word representations to the today's route.
Distributed word representations.
% TODO write with paper at hand
They showcased that word embeddings can indeed be used as ready made feature vectors and once trained, can be used for other applications by other researchers.
Important to note that they reported training times in the order of days, even weeks.

% \textcite{mikolov_distributed_2013} Word2Vec paper.
Word2Vec by \textcite{mikolov_distributed_2013} brought together the advancements and attractiveness that were brewing in the word embedding research.
First and foremost, they used an efficient loss function for their neural network architecture, the hierarchical softmax. % TODO ref here

With training time under manageable conditions \ldots.
Used negative subsampling, essentially a probability for a word to be discarded by inversely proportional to how frequent it is in the dataset.
Their most famous contribution is the quality of the vectors they have learned.
The theory set out by ? was empirically shown by \citeauthor{mikolov_distributed_2013} by demonstrating that countries and their capital cities exhibited a linear pattern on the PCA.
% TODO simplified graphic here, what is a PCA?

Also element-wise addition in section 5.
% TODO read section 5
They have been hosting their project open source but perhaps more importantly, they published an word2vec pretrained model on English on the internet.
Researchers and industry professionals have been using the embeddings since the semantic similarity between close words were relevant in numerous applications.
%Finally, word2vec lead up to \textcite{bajo}
% http://blog.aylien.com/overview-word-embeddings-history-word2vec-cbow-glove/
% Ruder says don't dwell on the specific embedding too much
% fasttext is trained on a huge corpus with tuned hyperparameters
% hence the choice of the specific model is not important


\section{Bilingual Word Embeddings}%
\label{sec:bilingual_word_embeddings}

\section{Document Retrieval}%
\label{sec:document_retrieval}

\section{Matching}%
\label{sec:matching}

\section{Approaches in Wordnet Generation}%
\label{sec:approaches_in_wordnet_generation}

% from Leveraging Parallel Corpora and Existing Wordnets for Automatic Construction of the Slovene Wordnet
WordNet generation is broken down into 4 categories
\begin{enumerate}
    \item Expand model, \textcite{vossen_introduction_1998}, fixed synsets are translated from English to target language.
    \item Link English entries from machine-readable bilingual dictionaries to English Princeton WordNet senses~\textcite{knight_building_1994}.
    \item Taxonomy parsing~\textcite{farreres_using_1998}.
    \item Ontology matching~\textcite{farreres_towards_2004}
\end{enumerate}

\textcite{gordeev_unsupervised_2018} uses unsupervised cross-lingual embeddings to match cross-lingual product classifications.
Working on taxonomy matching, they use out of domain pre-trained embeddings due to small size of their corpora and investigate methods using untranslated and translated text.

\textcite{lesk_automatic_1986} represent words using their gloss. Relied upon traditional dictionaries.
\textcite{banerjee_adapted_2002} developed on lesk algorithm and included WordNet definitions.
\textcite{khodak_automated_2017} used word embeddings and WordNet.

\textcite{metzler_similarity_2007} talked about short text retrieval and lexical matching. They reported that lexical matching is good for finding semantically identical matches.

\textcite{sagot_building_2008} built a French wordnet.

\textcite{xiao_distributed_2014} another embedding paper.

\textcite{kusner_word_2015} is Word Mover's Distance.

\textcite{balikas_cross-lingual_2018} suggested using optimal transport for cross-lingual document retrieval.

\textcite{arora_simple_2016} simple but tough-to-beat baseline for sentence embeddings.

\textcite{klementiev_inducing_2012} base paper for cross lingual word embeddings?

\textcite{irvine_comprehensive_2017} used as a guideline on best practices.

%\textcite{jonker_shortest_1987} lapjv paper.

% Chapter Template
\chapter{Unsupervised Matching}%
\label{chap:unsupervised_matching}

%%% ROUGH FIRST DRAFT

\section{Linear Assignment Using Sentence Embeddings}%
\label{sec:linear_assignment_using_sentence_embeddings}

Using word embeddings to obtain embeddings for longer pieces of text has been studied with implementations like doc2vec~\cite{le_distributed_2014} that builds upon the word2vec~\cite{mikolov_distributed_2013} model in order to learn paragraph embeddings.
However, there is an assumption of a continuous text for the given model.
When the text that we would like to show on a latent space is not part of a longer piece of text but \emph{discrete} pieces, that presumption does not hold.
With the dictionary definitions, we have such a case.
Our dictionary definitions are comprised of 10 to 11 words and there is no relation from one distinct dictionary definition to another.
% TODO reference the wordnet statistics table when you decide on where to put it
In other words, they are not continuous.
One other case where a similar situation occur is \emph{twitter}.
\emph{Tweets} are short pieces of text due to the 280 character constraint imposed by the platform.
With such short pieces of text, instead of paragraph embeddings, we can talk about \emph{sentence embeddings}.
A sentence embedding model should ideally capture the collective meaning of the short text where every word is potentially informative.

\textcite{zhao_ecnu_2015} used two approaches for SemEval-2015 Task 2: Semantic Textual Similarity~\footnote{\url{http://alt.qcri.org/semeval2015/task2/}}.
First, for a sentence $S = (w_{1}, w_{2}, \dots, w_{s})$ where the length of the presumably small sentence is $|S| = s$ and the word embedding of a $w_t$ is $v_t$;
\begin{itemize}
    \item They summed up the word embeddings of the sentence $\sum_{t \in S}v_{t}$
    \item Used information content~\cite{saric_takelab_2012} to weigh each word's LSA vector $\sum_{t \in S} I(w_t) v_{t}$
\end{itemize}
Both approaches results in a vector that is in the same dimensions $R^{d}$ as the original word representations.

\textcite{edilson_a._correa_nilc-usp_2017} expanded upon this simple yet effective idea to tackle the SemEval-2017 Task 4\footnote{\url{http://alt.qcri.org/semeval2017/task4}}, Sentiment Analysis in Twitter.
In order to acquire embeddings that represented \emph{tweets}, they weighed the word embeddings that made up a tweet; $\text{tweet}_i = (w_{i1}, w_{i2}, \dots, w_{im})$ with the \tfidf{} weights.
For the \tfidf{} calculation, they cast individual weights as documents so that term frequency become the term count in a single tweet while document frequency become the number of tweets the term $w_t$ occurs.

We have mentioned that our dictionary definitions are not continuous.
Yet, we advocate using \tfidf{} weights to weigh our word embeddings to get sentence embeddings.
In order to clarify, let us present Table~\ref{tab:en_it_examples}.

\noindent\fbox{%
    \parbox{\textwidth}{%
        % \caption{Example English Princeton WordNet definitions}
        turn red, as if in embarrassment or shame \\
        a feeling of extreme joy \\
        a person who charms others (usually by personal attractiveness) \\
        so as to appear worn and threadbare or dilapidated \\
        a large indefinite number \\
        distributed in portions (often equal) on the basis of a plan or purpose \\
        a lengthy rebuke
    }%
}%
% TODO maybe a fbox with more line width but captions don't work

\begin{table}
    \centering
    \caption{Some definitions from English Princeton WordNet}%
    \label{tab:en_it_examples}
    \begin{tabular}{l}
        \toprule
        turn red, as if in embarrassment or shame \\
        a feeling of extreme joy \\
        a person who charms others (usually by personal attractiveness) \\
        so as to appear worn and threadbare or dilapidated \\
        a large indefinite number \\
        distributed in portions (often equal) on the basis of a plan or purpose \\
        a lengthy rebuke \\
        \bottomrule
    \end{tabular}
\end{table}
% TODO this table is ugly can we do better?

% For sentence embeddings, first a \tfidf{} matrix is constructed.
For the \tfidf{} calculations, we followed a similar approach.
The term frequency is the raw count of a term in a dictionary definition.
While the document frequency is the number of dictionary definitions where $w_t$ occurs.

Then, with the term-embedding matrix at hand, we have calculated definition embeddings using;
\begin{equation}
    S_{\text{emb}}(S) = \sum_{w_{i} \in S} \varB{tf_{w_{i},S}-idf_{w_i}} \cdot Emb_{w}(w_{i})
\end{equation}
Every word that makes up a definition is scaled by its vector in ${\rm I\!R}^n$, then concatenated to form sentence embeddings on ${\rm I\!R}^n$.

% Bipartite graph or matrix
% Give the assumption, defintions are one-to-one
Given the N vectors from source and target language, we hypothesize that there exists a matching where every source definition vector is perfectly mapped to one target vector.
Given that this problem naively iterates over $N!$ matchings, we have looked into an algorithm.

%%% TODO lapjv %%%
% https://blog.sourced.tech/post/lapjv/
%
\section{Jonker Volgenant Algorithm}%
\label{sec:jonker_volgenant_algorithm}
% Ok jonker volgenant is super complicated
% can I just say I'm using linear assignment? maybe talk about hungarian algortihm a bit

\section{Results}%
\label{sec:results}

\begin{table}[htbp]
    \centering
    \begin{tabular}{lrrr}
        \toprule
& \multicolumn{3}{c}{Percentage of Correctly Matched Definitions} \\
\cmidrule(lr){2-4}
        \textbf{Language} & \textbf{fastText 1M} & \textbf{fastText 500k} & \textbf{Numberbatch} \\
        \midrule
        bg & 0.39 & 0.41 & 0.19 \\
        el & 0.37 & 0.38 & 0.14 \\
        it & 0.28 & 0.28 & 0.36 \\
        ro & 0.39 & 0.39 & 0.20 \\
        sl & 0.15 & 0.15 & 0.06 \\
        sq & 0.55 & 0.54 & 0.27 \\
        \bottomrule
    \end{tabular}
    \caption{Linear Assignment Using 2000 Definitions}%
    \label{tab:lapjv_2000}
\end{table}

\begin{table}[htbp]
    \centering
    \begin{tabular}{lrrr}
        \toprule
& \multicolumn{3}{c}{Percentage of Correctly Matched Definitions} \\
\cmidrule(lr){2-4}
        \textbf{Language} & \textbf{fastText 1M} & \textbf{fastText 500k} & \textbf{Numberbatch} \\
        bg & 0.35 & 0.36 & 0.18 \\
        el & 0.36 & 0.36 & 0.12 \\
        it & 0.25 & 0.25 & 0.32 \\
        ro & 0.36 & 0.37 & 0.19 \\
        sl & 0.11 & 0.11 & 0.05 \\
        sq & 0.39 & 0.40 & 0.19 \\
        \bottomrule
    \end{tabular}
    \caption{Linear Assignment Using 3000 Definitions}%
    \label{tab:lapjv_3000}
\end{table}

\begin{table}[htbp]
    \centering
    \begin{tabular}{lrrr}
        \toprule
& \textbf{fastText 1M} & \textbf{fastText 500k} & \textbf{Numberbatch} \\
\midrule
        Best & 0.55 & 0.54 & 0.36 \\
        Worst & 0.11 & 0.11 & 0.05 \\
        Average & 0.33 & 0.33 & 0.19 \\
        \bottomrule
    \end{tabular}
    \caption{Summary of Linear Assignment}%
    \label{tab:lapjv_summary}
\end{table}

% Chapter Template

\chapter{Dictionary Alignment as Pseudo-Document Retrieval}%
\label{chap:retrieval}

Document retrieval is the prototypical information retrieval task.
\textcite{bush_as_1945} first theorized the possibilities of the automatic information retrieval by machines in his essay titled \citetitle{bush_as_1945}.
\textcite{singhal_modern_2001} also gives due credit to \textcite{luhn_statistical_1957} for the suggestion of document retrieval using word overlap.

Modern information retrieval techniques are far from the scope of this thesis.
Considering the small collection of documents at hand, we will investigate if we can handle the task using approaches that were available to the researchers when the size of corpora that were available to them was small as well~\cite{singhal_modern_2001}.
However, we will get leverage from a state of the art tool from the modern computer science that is Google Translate.

\section{Machine Translation}

The first method we will study starts off by translating the target language's corpora to English using Google Translate.
We have used the Google Cloud API\footnote{\url{https://cloud.google.com/translate}} in order to automate the process.
% TODO when the preparing the wordnets section is ready, refer there from here

With the English Princeton WordNet definitions and the target wordnet definitions at hand, we can handle the task as monolingual document retrieval.
We have used the vector space representation we have talked about in Chapter~\ref{chap:background_n_related}.



We have chosen \tfidf{} as to ask if the task at hand can be solved by naive tools.
In order to get \tfidf{} scores of the documents, first a term-document matrix is created.
Documents being definitions and with an average of 10.62 words per definition, the resulting matrix is parse.
In a \tfidf{} matrix, for an entry in the matrix $w_{i,j}$, we can give the formula for it as:
\begin{equation*}
    \varB{tf_{w,d}-idf_{w}} = {\sum_{w' \in d}{f_{w',d}}} \cdot \log \frac {N} {df_w}
\end{equation*}
Such that term $w_{i,j}$ depicts the importance of term $t$ with relation to its general importance throughout the corpus.
Now we can define the similarity between the documents as the cosine similarity between their \tfidf{} vectors.
For the row $w_t$ and $w_p$, cosine similarity between definitions $t, p$ is
\begin{equation*}
    \cos(\theta) =
\end{equation*}

Definitions are then separated into queries and corpora.
Query definitions is then matched up against every definition in the corpora and the ten documents that are closest in terms of cosine similarity is retrieved.
Within the retrieved documents, if the document with the matched sense id is retrieved in the first result, this is taken as a hit at 1.
Mean Reciprocal Rank is also calculated in order to show the success of a retrieval scenario.

Where monolingual retrieval falls short, we leveraged the power of word embeddings to capture the semantic information of the words.
A famous example for the inadequacy of \tfidf{} is illustrated by~\cite{kusner_word_2015}.
For two snippets of text; \emph{Obama speaks to the media in Illinois} and \emph{The President greets the press in Chicago} Kusner argues that while they convey the same information, they would be near orthogonal in a bag of words setting.
Yet before moving forward with WMD, we wanted to test sentence embeddings.

\section{Cross Lingual Document Retrival}%
\label{sec:cross_lingual_document_retrival}

\subsection{Optimal Transport}%
\label{sub:optimal_transport}

\subsection{Sinkhorn}%
\label{sub:sinkhorn}







% Chapter Template

\chapter{Supervised Validation}%
\label{chap:supervised_validation}

The approaches we have presented so far work fully unsupervised.
Given two collections of definitions, we have tried to retrieve the definition(s) that represented the same meaning.
However, given the moderately sized data in our hands we have accepted as \enquote{golden}, we studied the feasibility of training an encoder~\cite{sutskever_sequence_2014}, where the objective is to learn whether the pair of definitions entail the same sense across languages.

Recurrent Neural Network (RNN) architectures improve upon the prototypical neural network model by introducing a \emph{memory} for the connections in the network~\cite{rumelhart_learning_1986}.
By updating the hidden unit over time using the output of the previous time-step, the model can \emph{remember} features of the input signal for later inputs.
% TODO figure here
%RNN seemed fitting for our case given their success on tasks that require a model to \emph{remember} previous signals over a long sequence of inputs\cite{gers_lstm_2001,jean_using_2014}.
One particular architecture of RNNs propsed by \textcite{hochreiter_long_1997} is \emph{long short-term memory} (LSTM).
LSTM models have been successful on language modelling tasks~\cite{sutskever_sequence_2014}, handwriting recognition~\cite{graves_unconstrained_2008,graves_novel_2009} and machine translation with a focus on rare words~\cite{luong_addressing_2014} to name a few.
Highlight of these results are that LSTM has an advantage on tasks that require contextual information to persist over long periods of time~\cite{graves_long_2012}.
Furthermore, LSTMs do not require fixed input vectors, which is a necessity for us since our definition pairs do not have to be the same size.

\subsection{Vanishing Gradient Problem}%
\label{sub:vanishing_gradient_problem}

LSTM is borne out of the need to address the \emph{vanishing gradient problem}~\cite{hochreiter_long_1997, bengio_learning_1994}.
On the original publication by \textcite{hochreiter_long_1997}, a crucial shortcoming of RNNs have been identified as their slow rate of training which may not converge in the end at all.
Independently, \textcite{bengio_learning_1994} suggested that the problem stems from the choice between the conservation of the previous inputs versus resisting against the noise they accumulate.
Figure~\ref{fig:vanishing_gradients}, adapted from \textcite{graves_long_2012} illustrates the problem using shades as the influence of input over neural network units.

\begin{figure}[htbp]
    \centering
    \includegraphics[page=1,width=\textwidth]{Figures/outputs_hiddenlayer.pdf}
    \caption{Graphical representation of vanishing gradient problem where the shades of the nodes represent the influence of the input signal, adapted from Figure 4.1 of \textcite{graves_long_2012}}%
    \label{fig:vanishing_gradients}
\end{figure}

% Input -> will cell learn/change
% Forget -> will cell reset
% Output -> Will cell propagate

%The issue is related to a natural disadvantage of Recurrent Neural Networks (RNN).
As the input signal traverses the units of an RNN, it either diminishes or blows up~\cite{graves_long_2012}.
A long line of solutions were proposed such as decomposing structures so that only unexpected ones can be relevant.
% TODO give more solutions here
LSTM is the solution highlighted by \textcite{graves_long_2012} as a recurrent neural network model that can work over temporally distant input signals while preserving their influence or diminishing their noise.
The centrepiece idea is to use a \emph{constant error carousel}, special cells that enforce a constant error flow.
In the original paper, this complex unit is named \emph{memory cell}~\cite{hochreiter_long_1997}.
Using an \emph{input gate}, the cell is updated if the current input is relevant and using an \emph{output gate}, the unit will not update other cells if the current input is not relevant.
A simplified overview of the suggested model is presented in Figure~\ref{fig:simple_lstm}.

\begin{figure}[htbp]
    \centering
    \incfig{simple_lstm}
    \caption{Simplified long short-term memory cell architecture}%
    \label{fig:simple_lstm}
\end{figure}
% We kept forward pass and backprop equations out because no time and out of scope hopefully

Multiple arrows denote input from current time frame and recurrent connections.
$\odot$ symbol denotes multiplication.
By weighing the input and output gates between 0 and 1, the impact of the current input can be adjusted.
Overall, input gate controls how much cell will learn and output gate controls how much the cell will propagate.
Figure~\ref{fig:lstm_preserves} adapted from \textcite{graves_long_2012} illustrates how the LSTMs operate against vanishing gradient.

\begin{figure}[htbp]
    \centering
    \includegraphics[page=1,width=\textwidth]{Figures/outputs_hiddenlayer2.pdf}
    \caption{Preserving the input signal through blocking (-) or allowing (O) the input signal, adapted from Figure 4.4 of \textcite{graves_long_2012}}%
    \label{fig:lstm_preserves}
\end{figure}

% forget gate
Two gates on top of the cell structure model got extended with a third \emph{forget gate} by \textcite{gers_learning_2000}.
The aim was to handle input sequences that are not segmented in a predictable manner.
The proposed forget gate is implemented to \emph{reset} the cell.
When a cell state got irrelevant due to a change in problem domain, forget gate gradually resets the cell state instead of erroneous activations from the input gate.

% peephole connections
Another extension came in the form of \emph{peephole connections} by \textcite{gers_learning_2003}.
By allowing internal gates to inspect the cell state, they have shown improvements on non-linear tasks.

% \textcite{pascanu_difficulty_2012}

The finalized model with input, forget and output gates as well as the internal peephole connections was debuted in \textcite{graves_framewise_2005}.
In their expansive study comparing 8 LSTM variants over 15 years of CPU time, \textcite{greff_lstm_2017} named this model the \enquote{vanilla LSTM}.

Recently, \textcite{mueller_siamese_2016} proposed a \emph{siamese} model using two LSTMs.

% Optimizer adelta
\textcite{zeiler_adadelta_2012}

% Mean squared error loss

% https://keras.io/layers/recurrent/ implementation details here

% Chapter Template

\chapter{Experiments and Evaluation}%
\label{chap:experiments_and_evaluation}

\section{Preparing Wordnets}%
\label{sec:preparing_wordnets}

In order to run our experiments we need two sets of dictionary definitions from two different languages.
Open Multilingual Wordnet~\cite{bond_linking_2013} project hosts 34 wordnets with permissive licenses on their website.\footnote{\url{http://compling.hss.ntu.edu.sg/omw/}}
We have investigated the available wordnets and fortunately six of them included definitions, also known as glosses.
Since we do not use any other information related to wordnets (like semantic relationships) only definitions were extracted into a plain text corpora.
This intermediate corpora includes WordNet 3.0 synsets identifiers and the corresponding definitions in the target language.
Natural Language Toolkit~\cite{bird_natural_2009} provides an API for reading and retrieving English Princeton WordNet.
Using the synsets identifiers, it is possible to retrieve the exact synset which comes attached with an unique definition for the synset.
Finally, we have 6 aligned corpora for 6 wordnets we will run the experiments on.
Alignment here refers to definitions that represent same synset across languages appearing on the same index, a notation we will use throughout the chapter.
The statistics and the 2 letter language codes that we will commonly use to denote the wordnets for the rest of this chapter is presented in Table~\ref{tab:wordnet_stats}.

\begin{table}[hbtp]
    \centering
    \settowidth\tymin{\textbf{Language}}
    \setlength\extrarowheight{2pt}
    \begin{tabulary}{1.0\linewidth}{L L R R R R}
        \toprule
        Language Code & Language Name & Number of Definitions & Number of words & Average Words per Definition & Longest Definition \\ \midrule
        sq & Albanian & 4681 & 54980 & 11.75 & 101 \\
        bg & Bulgarian & 4959 & 63014 & 12.71 & 53 \\
        el & Greek & 18136 & 203924 & 11.24 & 89 \\
        it & Italian & 12688 & 93005 & 7.33 & 35 \\
        ro & Romanian & 58754 & 586304 & 9.98 & 105 \\
        sl & Slovene & 3144 & 39865 & 12.68 & 68 \\
        \bottomrule
    \end{tabulary}%
    \caption{Language codes and statistics for the target wordnets used in the thesis.}%
    \label{tab:wordnet_stats}
\end{table}

\section{Preparing Word Embeddings}%
\label{sec:preparing_word_embeddings}

In Chapter~\ref{chap:background_n_related}, we have mentioned the recent popularization of pre-trained word embeddings.
Initiated by word2vec\footnote{\url{https://code.google.com/archive/p/word2vec/}}, other sources for word embeddings are GloVe\footnote{\url{https://nlp.stanford.edu/projects/glove/}}, fastText\footnote{\url{https://fasttext.cc/}} and numberbatch\footnote{\url{https://github.com/commonsense/conceptnet-numberbatch}}.
Yet, word embeddings for languages other than English are scarce.
FastText hosts word embeddings for 157 languages so we used them as our primary source.\cite{grave_learning_2018}
These embeddings are trained using Common Crawl and Wikipedia data.

Numberbatch provides word embeddings for 304 languages.
However, 10 of the supported languages are presented as core languages with excellent support and 68 of them are tagged as common languages which is only given adequate support.
Referring to Table~\ref{tab:wordnet_stats}, Italian is among the core languages and the rest are in the common languages group.

The fastText embeddings include 2 million tokens out of the box.
In order to increase the efficient of the experiments, we have cut down the size of the embeddings into 1 million and 500 thousand.
The vectors are sorted according to their corpus frequency so the uppermost lines were used.

Numberbatch embeddings are distributed in one large file and does not include fixed number of tokens per language.
Hence after parsing the file and extracting the embeddings into individual files, the number of word vectors we are left with are presented in Table~\ref{tab:numberbatch_stats}.

\begin{table}[hbtp]
    \centering
    \begin{tabulary}{1.0\linewidth}{L R}
        \toprule
        Language Code & Number of Tokens \\
        \midrule
        bg & 20871 \\
        el & 16926 \\
        en & 417195 \\
        it & 91829 \\
        ro & 10874 \\
        sl & 11458 \\
        sq & 5512 \\
        \bottomrule
    \end{tabulary}
    \caption{The number of embeddings available in numberbatch}%
    \label{tab:numberbatch_stats}
\end{table}

These embeddings are monolingual, so they are on separate arbitrary latent spaces.
In order to represent them on the same latent space, we used VecMap~\cite{artetxe_robust_2018,artetxe_generalizing_2018,artetxe_learning_2017,artetxe_learning_2016}.\footnote{\url{https://github.com/artetxem/vecmap}}
According to \textcite{ruder_survey_2017}, bilingual or cross lingual embedding models optimize similar objectives and differences in performance is due to available data they are trained on.
\textcite{glavas_how_2019} supports this intuition and has empirically proven that common evaluation metrics like bilingual dictionary induction is not representative for the bilingual embedding's performance on downstream tasks.
Hence our preference of VecMap is highly influenced by it's availability as an open source framework and it's ease of training.

Best performing VecMap model available in the framework is supervised alignment.
It requires a bilingual dictionary, otherwise known as aligned word pairs for two languages.
We sourced our bilingual dictionary from Open Subtitles 2018\footnote{\url{http://www.opensubtitles.org/}} data as hosted by OPUS.\footnote{\url{http://opus.nlpl.eu/}}
The dictionary can be sorted by the confidence score of the translation pair, which we did so that pairs with high confidence scored swam to top.
The first 25000 translation pairs were shuffled and split into training and testing examples for 6 languages.
After supervised mapping of language specific word embedding and English word embedding, we have 6 pairs of vectors that share the same latent space.

\begin{table}[htbp]
    \centering
    \begin{tabular}{lrrr}
        \toprule
        \textbf{Language} & \textbf{FastText 1M} & \textbf{FastText 500k} & \textbf{numberbatch} \\
        \midrule
        bg & 33.61 & 35.17 & 51.97 \\
        el & 37.37 & 39.58 & 30.35 \\
        it & 58.20 & 59.28 & 50.37 \\
        ro & 37.33 & 38.71 & 64.17 \\
        sl & 21.42 & 22.91 & 74.74 \\
        sq & 24.46 & 25.36 & 58.63 \\
        \bottomrule
    \end{tabular}
    \caption{Accuracy scores (in percentage) of the word embeddings aligned using VecMap}%
    \label{tab:accuracy_results}
\end{table}

\begin{table}[htbp]
    \centering
    \begin{tabular}{lrrr}
        \toprule
        \textbf{Language} & \textbf{FastText 1M} & \textbf{FastText 500k} & \textbf{numberbatch} \\
        \midrule
        bg & 96.43 & 93.36 & 17.53 \\
        el & 94.44 & 90.28 & 12.15 \\
        it & 97.93 & 95.97 & 41.08 \\
        ro & 97.06 & 94.91 & 16.4 \\
        sl & 94.67 & 90.73 & 9.23 \\
        sq & 83.59 & 80.92 & 9.51 \\
        \bottomrule
    \end{tabular}
    \caption{Coverage scores (in percentage) of the word embeddings aligned using VecMap}%
    \label{tab:coverage_results}
\end{table}

Even though bilingual dictionary induction is not representative of a bilingual word embedding pair's performance on downstream tasks~\cite{ruder_survey_2017,glavas_how_2019}, we include the evaluation results obtained from VecMap framework as a quick measure for their quality on Table~\ref{tab:accuracy_results} for accuracy and Table~\ref{tab:coverage_results} for coverage.
Accuracy is the measure for correctly identifying the translation of a word given the test dictionary and coverage is the percentage of translation pairs that could be inducted.
From the results, it is apparent that fastText embeddings have much better coverage due to vast data they were trained on.
Yet, numberbatch exhibits better accuracy scores which might be a tradeoff of their low coverage.
Nevertheless, bilingual dictionary induction results are not indicative of word embedding's real life performance.

\section{Results}%
\label{sec:results}

In this section, we will present the results of our experiments in the order they appeared in the main text.

\subsection{Matching Results}%
\label{sub:matching_results}

In order to evaluate the definition matching of sentence embeddings that we presented in Chapter~\ref{chap:unsupervised_matching}, we choose 3 data sizes to experiment on; 2000 definitions, 3000 definitions and all available definitions each.
As mentioned before, since definition matching is a one-to-one operation, we can only report the accuracy of the one result matched per definition.

\begin{table}[htbp]
    \centering
    \begin{tabular}{lrrr}
        \toprule
& \multicolumn{3}{c}{Percentage of Correctly Matched Definitions} \\
\cmidrule(lr){2-4}
        \textbf{Language Code} & \textbf{fastText 1M} & \textbf{fastText 500k} & \textbf{Numberbatch} \\
        \midrule
        bg & 39.35 & 40.75 & 19.00 \\
        el & 36.90 & 37.70 & 14.35 \\
        it & 27.70 & 28.25 & 36.30 \\
        ro & 38.65 & 39.45 & 20.25 \\
        sl & 14.50 & 15.05 & 5.80 \\
        sq & 54.85 & 54.15 & 27.05 \\
        \bottomrule
    \end{tabular}
    \caption{Definition matching evaluated on 2000 definition pairs}%
    \label{tab:lapjv_2000}
\end{table}

\begin{table}[htbp]
    \centering
    \begin{tabular}{lrrr}
        \toprule
& \multicolumn{3}{c}{Percentage of Correctly Matched Definitions} \\
\cmidrule(lr){2-4}
        \textbf{Language Code} & \textbf{fastText 1M} & \textbf{fastText 500k} & \textbf{Numberbatch} \\
        \midrule
        bg & 35.27 & 36.00 & 18.10 \\
        el & 36.13 & 36.07 & 11.70 \\
        it & 24.67 & 24.90 & 32.07 \\
        ro & 36.43 & 36.87 & 18.73 \\
        sl & 11.27 & 11.40 & 4.63 \\
        sq & 39.43 & 39.67 & 19.03 \\
        \bottomrule
    \end{tabular}
    \caption{Definition matching evaluated on 3000 definition pairs}%
    \label{tab:lapjv_3000}
\end{table}

\begin{table}[htbp]
    \centering
    \begin{tabular}{@{}lrrr@{}}
        \toprule
 & \multicolumn{3}{l}{Percentage of Correctly Matched Definitions} \\ \cmidrule(l){2-4}
 & \textbf{fastText 1M} & \textbf{fastText 500k} & \textbf{Numberbatch} \\ \midrule
        \textbf{Best} & 54.85 & 54.15 & 36.30 \\
        \textbf{Worst} & 11.27 & 11.40 & 4.63 \\
        \textbf{Average} & 32.93 & 33.35 & 18.92 \\
        \bottomrule
    \end{tabular}
    \caption{The summary of matching dictionary definitions}%
    \label{tab:lapjv_summary}
\end{table}

In order to assess the feasibility of this algorithm in a real life data, we ran the experiment using all of the available dictionary definitions.
We present the results in Table~\ref{tab:lapjv_full}.

Romanian wordnet definitions did not fit into the memory of the machine we tested on (64 GB of available RAM) so we could not present the full definition matching results for Romanian wordnet.

\begin{table}[htbp]
    \centering
    \begin{tabular}{@{}lrr@{}}
        \toprule
        \textbf{Language Code} & \textbf{Definitions} & \textbf{Percentage of Definitions Matched Correctly} \\ \midrule
        bg & 4958 & 28.52 \\
        el & 18105 & 31.06 \\
        it & 12585 & 15.65 \\
        sl & 3143 & 11.42 \\
        sq & 4680 & 44.81 \\
        \bottomrule
    \end{tabular}
    \caption{Definition matching using all the available definition in the corpora}%
    \label{tab:lapjv_full}
\end{table}

\subsection{Monolingual Retrieval Results}%
\label{sub:chap4_results}

We have presented our method of translating the target wordnet definitions into English and running monolingual document retrieval on the corpora in Section~\ref{sec:monolingual_retrieval}.
As mentioned before, we handled the original English definitions as the document collection to retrieve from and used the translated target wordnet definitions as queries.
We truncated the definitions to 2000 pairs each and ran the experiment.

The results are presented in Table~\ref{tab:monolingual_tfidf}.
Considering the other approaches, \tfidf{} weighted term document matrices and cosine similarity between definitions does not seem adequate for solving the task at hand.
Mean reciprocal rank penalizes the query according to the rank of the correct result, if the correct definition is retrieved on a lower rank, the MRR diminishes.
The divide between the MRR score and the precision at 1 score indicates that for some definitions the noise that arises from the translation keeps the correct definition from appearing among the top results heavily.

\begin{table}[htbp]
    \centering
    \begin{tabulary}{1.0\linewidth}{L R R R R R}
        \toprule%
        \textbf{Language Code} & \textbf{MRR} & \textbf{Precision at 10} & \textbf{Precision at 10 \%} & \textbf{Precision at 1} & \textbf{Precision at 1 \%} \\
        \midrule%
        bg & 0.087 & 674 & 0.337 & 403 & 0.202 \\
        el & 0.122 & 1006 & 0.503 & 709 & 0.355 \\
        it & 0.016 & 476 & 0.238 & 250 & 0.125 \\
        ro & 0.051 & 988 & 0.494 & 728 & 0.364 \\
        sl & 0.073 & 584 & 0.292 & 317 & 0.159 \\
        sq & 0.056 & 979 & 0.490 & 767 & 0.384 \\
        \bottomrule
    \end{tabulary}
    \caption{Experiment results for monolingual retrieval, ran on 2000 definition pairs}%
    \label{tab:monolingual_tfidf}
\end{table}

\subsection{Cross Lingual Document Retrieval Results}%
\label{sub:cross_lingual_retrieval_results}

We present the results of the method we have presented in Section~\ref{sec:cross_lingual_document_retrival} on the Table~\ref{tab:cldr_results}.
Due to space restrictions, we omitted the language code header and used \enquote{\%} as the percentage of precision at one metric identifier.

\begin{landscape}
\begin{table}[htbp]
\centering
\begin{tabular}{llrrrrrrrrr}
    \toprule
 &  &  & \multicolumn{2}{r}{\textbf{WMD tf-idf}} & \multicolumn{2}{r}{\textbf{Sinkhorn tf-idf}} & \multicolumn{2}{c}{\textbf{WMD tf}} & \multicolumn{2}{c}{\textbf{Sinkhorn tf}} \\ \cline{4-11}
 & \textbf{Vector} & \textbf{Vocabulary} & \multicolumn{1}{c}{\textbf{MRR}} & \multicolumn{1}{c}{\textbf{\%}} & \multicolumn{1}{c}{\textbf{MRR}} & \multicolumn{1}{c}{\textbf{\%}} & \multicolumn{1}{c}{\textbf{MRR}} & \multicolumn{1}{c}{\textbf{\%}} & \multicolumn{1}{c}{\textbf{MRR}} & \multicolumn{1}{c}{\textbf{\%}} \\ \cline{2-11}
\multirow{3}{*}{bg} & 1M FastText & 10532 & 50.83 & 41.50 & 51.85 & 42.65 & 41.81 & 33.95 & 42.76 & 34.60 \\
 & 500k FastText & 10457 & 51.15 & 41.90 & 52.24 & 43.00 & 42.18 & 34.15 & 43.19 & 34.80 \\
 & Numberbatch & 6028 & 25.32 & 17.45 & 24.97 & 17.15 & 18.55 & 12.30 & 18.09 & 11.35 \\ \cmidrule(lr){2-11}
\multirow{3}{*}{el} & 1M FastText & 9307 & 47.82 & 39.05 & 48.67 & 40.15 & 40.74 & 32.85 & 41.42 & 33.40 \\
 & 500k FastText & 9168 & 47.45 & 38.45 & 48.45 & 39.95 & 40.53 & 32.55 & 41.39 & 33.55 \\
 & Numberbatch & 5127 & 19.92 & 13.15 & 20.06 & 13.40 & 14.71 & 9.95 & 14.94 & 10.00 \\ \cmidrule(lr){2-11}
\multirow{3}{*}{it} & 1M FastText & 10025 & 40.24 & 31.15 & 40.60 & 31.45 & 31.98 & 23.50 & 32.07 & 23.40 \\
 & 500k FastText & 9975 & 40.27 & 31.15 & 40.49 & 31.30 & 32.11 & 23.65 & 32.21 & 23.50 \\
 & Numberbatch & 8875 & 42.72 & 33.30 & 42.77 & 33.35 & 35.11 & 26.70 & 35.12 & 26.80 \\ \cmidrule(lr){2-11}
\multirow{3}{*}{ro} & 1M FastText & 12165 & 51.30 & 41.60 & 51.95 & 41.50 & 44.20 & 35.90 & 45.06 & 35.60 \\
 & 500k FastText & 12034 & 51.28 & 41.65 & 52.37 & 42.20 & 43.85 & 35.50 & 45.14 & 35.75 \\
 & Numberbatch & 6939 & 27.68 & 19.85 & 27.70 & 19.80 & 21.57 & 15.70 & 21.86 & 16.25 \\ \cmidrule(lr){2-11}
\multirow{3}{*}{sl} & 1M FastText & 12185 & 26.22 & 17.80 & 26.38 & 18.05 & 23.43 & 15.10 & 23.97 & 15.60 \\
 & 500k FastText & 12020 & 26.12 & 17.80 & 26.31 & 17.95 & 23.47 & 15.15 & 24.03 & 15.80 \\
 & Numberbatch & 5870 & 9.35 & 4.80 & 9.46 & 5.00 & 6.35 & 2.90 & 6.26 & 2.85 \\ \cmidrule(lr){2-11}
\multirow{3}{*}{sq} & 1M FastText & 8048 & 65.66 & 59.15 & 63.97 & 56.40 & 56.47 & 48.55 & 56.94 & 49.30 \\
 & 500k FastText & 7990 & 65.61 & 58.70 & 64.42 & 56.85 & 56.61 & 48.80 & 57.05 & 49.25 \\
 & Numberbatch & 4908 & 31.07 & 23.55 & 31.06 & 23.30 & 24.31 & 17.80 & 24.74 & 18.35 \\
 \bottomrule
\end{tabular}
\caption{Results of cross lingual document retrieval cast as dictionary alignment}%
\label{tab:cldr_results}
\end{table}
\end{landscape}

%%%%%%%%%%%%%%%%%%%%%%%%%%
%  IN DOMAIN EMBEDDINGS  %
%%%%%%%%%%%%%%%%%%%%%%%%%%
\section{Investigating Word Embedding Sources}%
\label{sec:investigating_word_embedding_sources}

We experimented with in domain \emph{fastText} embeddings in order to ask if pre-trained embeddings are better than embeddings trained on the experiment data.
Since Romanian wordnet has the most data available, we have trained Romanian embeddings on the Romanian wordnet definitions, mapped the embeddings to the same latent space using supervised VecMap and ran cross lingual document retrieval experiments using word mover's distance and Sinkhorn.
% TODO here
The performance dropped to the quarter of pre-trained embeddings so we have not repeated the experiments for other language corpora.


%%%%%%%%%%%%%%%%
%  CASE STUDY  %
%%%%%%%%%%%%%%%%
\section{Case Study}%
\label{sec:case_study}

In order to test our approach, we have acquired a general purpose Turkish dictionary~\footnote{We would like to thank Hülya Küçükaras for providing the dictionary}, to be used for research purposes.
The dictionary was in a proprietary format so we have parsed the terms alongside their definitions.
The parts of speech for the terms are also extracted.
All in all, 67351 headwords with 93062 definitions are parsed.

We have shown that the approaches we have presented so far are bound by their memory restrictions.
We have tried to overcome it by running the experiment on only nouns but the issue persisted for the case study as well.
As a result, we referred to \textcite{khodak_automated_2017} and constrained our scope to a list of core WordNet synsets.
Open Multilingual Wordnet hosts~\footnote{\url{http://compling.hss.ntu.edu.sg/omw/wn30-core-synsets.tab}} a list that identifies 4961 WordNet identifiers in the form of offset and part of speech that is compatible with the nltk library.
The list has been prepared in \textcite{boyd-graber_adding_2006} by human evaluators by selecting salient synsets from a list of frequent words.
By using a core WordNet, we picked ourselves a problem domain we can tackle.
We also deleted the identifiers for verbs and adjectives and worked only with nouns.
Then, the experiment set for the Turkish dictionary is selected by translating the lemmas that belong to core WordNet synsets to Turkish and using the resulting set to query the headwords of the Turkish dictionary.
Using this method, we obtained 702 Turkish definitions and 3280 WordNet definitions.

The methods we studied so far work with two corpora of the same size so we randomly picked 702 English WordNet definitions to go against 702 Turkish dictionary definitions.
We use our best performing set of approaches; word mover's distance using \tfidf{} weights, run on fastText embeddings mapped using VecMap.
The bilingual dictionary provided by OpenSubtitles used in order to map Turkish and English fastText embeddings.

While preparing the corpora for the word mover's distance, 101 Turkish definitions are dropped due to them having no words to be represented by fastText embeddings.
Same number of English WordNet definitions are also dropped to keep to the symmetric size constraint.
Then, word mover's distance is run over 601 Turkish and English definitions.

We do not know the ground truth for the mapping in this study hence we cannot report for any evaluation metric.
As a result, in Appendix~\ref{app:case_study}, we present 100 randomly selected pairs that were retrieved as the top result against the Turkish query.

% Chapter Template

\chapter{Conclusion}%
\label{chap:conclusion}

In this study, we set out to investigate the feasibility of representing senses using their dictionary definitions.
Along the way, we used document retrieval, linear programming and neural networks to answer the issue on as many angles as possible.
The grand aim of the study was to compare the approaches that we had identified for the task.
To our best knowledge, a comparable study where the dictionary alignment approaches were reported on the basis of their performance is not available so we had to anchor the study to itself.
At the end of the day, we can make justified comparisons.

The monolingual retrieval using \tfidf weights and cosine similarity measure was chosen as a baseline because it is the most greedy approach available.
If dictionary generation could be solved by automatic machine translation, this thesis would not take hold.
The results presented in Chapter~\ref{chap:experiments_and_evaluation} prove so.

The matching algorithm is interesting.
Moving on with our greedy connotation, for a task like dictionary alignment, assigning a sense to a definition that is closest to it by some distance metric might leave another definition with less than an ideal match later down the line.
Matching ensures that the \emph{closest} metric in between definitions holds not just for individual definitions but for the whole corpora.
We can refer to Figure~\ref{fig:bipartite_graph} to illustrate this point.

We have mentioned the lexical gap problem in Chapter~\ref{chap:background_n_related} where some senses do not have equivalences in the target language.
Recently, \textcite{bolukbasi_man_2016} reported on gender biases of word embedding models which numberbatch embeddings responded with so called de-biased embeddings, eliminating it from their models almost completely.
Considering the most common type of lexical gap arises from languages with grammatical gender, possible effect of this on the matching approach is left for future work.

Overall, matching approach consistently shown the best performance across the board, supporting our hypothesis that one-to-one matching two sets of dictionary definitions would result in superior performance.
We have also proven our justification behind the choice of the particular embedding model and the fact that conventional evaluation of word embeddings might not translate to downstream tasks.
Numberbatch has scored first place on SemEval-2017 Task 2~\cite{camachocollados-EtAl:2017:SemEval}, on multilingual word similarity task.
Yet, against fastText embeddings, their model performed worse with the exception of Italian.
Italian is a \emph{core language} for numberbatch, where they claim full support.
It is also the language where numberbatch consistently outperformed fastText embeddings.
We have set out to investigate the effect of particular choices like this for the dictionary alignment task.
It can be reported with confidence that the advantage of one embedding model over another is not clear cut and should be investigated further.

With the supervised long short-term memory approach, we have observed that not only it is possible to represent senses using their dictionary definitions but also the metric of \emph{representing} the same sense can be learned.
The data required for obtaining any good performance should be noted and experimenting on diverse data should be left for future work.

The crucial shortcoming is the data requirements we have.
On one hand, any type of description that represent a sense can be aligned not with just WordNet but any dictionary.
Projects like BabelNet\footnote{\url{https://babelnet.org}} or ConceptNet are creating semantic databases of their own while WordNet is on version 3.0, still online well after 20 years.
Natural language processing research relies on external sources of information and the pre-annotated nature of these resources will always find a use.
Working towards automatically extending them creates more opportunities for sprawling research later down the line.

Our main contribution in this study is the empirical comparison of alignment and retrieval approaches.
We have hypothesized that aligning definitions one-to-one instead of greedily assigning each definition to it's closest counterpart will perform better.
Our intuition behind the hypothesis is that dictionaries include discrete senses.
Once a pair of definitions is matched, continuing to align further senses to any of the definitions can only deteriorate the performance.
The results we have presented in Section~\ref{sec:case_study} confirms our hypothesis.
Matching approaches outperformed retrieval approaches on any language set.
Including 6 different languages and observing the performance differences on all of them further confirms that by using the power of word embeddings, our finding are as language agnostic as possible.
Our final conclusion is that the state of the art approach Sinkhorn distance~\cite{balikasCrosslingual2018} between term document representation outperformed sentence embeddings that were proposed specifically for short text representation.
Further studies in the field can take this finding into account in their models.

\section{Future Work}%
\label{sec:future_work}

Throughout the thesis, English was always the centrepiece of the experiments.
The wordnets were evaluated by their alignment towards the first and the most comprehensive, WordNet.
The word embeddings were mapped to share a latent space with English word embeddings.
As we have mentioned in Chapter~\ref{chap:background_n_related}, ideas like Inter-Lingual Index offer ways to bypass the English as a hub language.
As an immediate future work, alignments that do not use English nor English Princeton WordNet can be investigated.
Culturally or syntactically closer languages can be bridges more easily than distant yet abundant English.

Recent transfer learning models like BERT~\cite{devlin_bert_2018} offer a novel way to overcome the fundamental shortcoming with the supervised encoder we presented;
the model performs in accordance with the available data and requires aligned data to function in the first place.
Transfer learning inspires approaches like encoding the metric for representing the same sense in $n$ languages after which the model is ready to predict on $n+1^{th}$ language.
Very recently, \textcite{jawanpuriaLearning2019} proposed a VecMap like framework for convenient alignment of word embeddings.
To our interest, the framework can map \emph{multilingual} embeddings on a \emph{shared space}.
With a potential synset discovery approach like the one proposed by \textcite{ruizcasadoAutomatic2005} where possible sense definitions are found and validated using supervised learning will be investigated next using the novel ideas as inspiration.


%----------------------------------------------------------------------------------------
%	THESIS CONTENT - APPENDICES
%----------------------------------------------------------------------------------------

\appendix % Cue to tell LaTeX that the following "chapters" are Appendices

% Include the appendices of the thesis as separate files from the Appendices folder
% Uncomment the lines as you write the Appendices

\chapter{Case Study - Aligning a Turkish Dictionary and English Princeton WordNet}%
\label{app:case_study}
\begin{longtable}{p{.50\textwidth} p{.50\textwidth}}
    \toprule
    \textbf{Turkish Definition} & \textbf{English WordNet Definition} \\
    \midrule
    \endfirsthead
    \multicolumn{2}{c}%
    {\tablename\ \thetable\ -- \textit{Continued from previous page}} \\
    \midrule
    \textbf{Turkish Definition} & \textbf{English WordNet Definition} \\
    \midrule
    \endhead
    \multicolumn{2}{r}{\textit{Continued on next page}} \\
    \endfoot
    \midrule
    \endlastfoot
    birdenbire duyulan ağrı ya da türlü heyecanları anlatır & an absence of emotion or enthusiasm \\
    \cmidrule(rl){1-2}
    meyve bisküvi vb ile yapılan bir i̇ngiliz tatlısı & a small hard fruit \\
    \cmidrule(rl){1-2}
    yalıtım tecrit & a thin coating or layer \\
    \cmidrule(rl){1-2}
    bulgu araz & a pattern of symptoms indicative of some disease \\
    \cmidrule(rl){1-2}
    42195 m’lik en uzun yaya koşusu uzunkoşu & any long ditch cut in the ground \\
    \cmidrule(rl){1-2}
    benzer olma durumu müşabehet & the quality of being similar \\
    \cmidrule(rl){1-2}
    var olan bulunan & something that is lost \\
    \cmidrule(rl){1-2}
    parmakların kapanmasıyla elin aldığı biçim & handwear covers the hand and wrist \\
    \cmidrule(rl){1-2}
    zayıf olma durumu & any strong feeling \\
    \cmidrule(rl){1-2}
    bir süreç içindeki durumlardan her biri basamak aşama rütbe mertebe & the effect of one thing or person on another \\
    \cmidrule(rl){1-2}
    bir birimin bölündüğü eşit parçalardan birini ya da birkaçını anlatan sayı & a unit of spoken language larger than a phoneme \\
    \cmidrule(rl){1-2}
    haşhaş sütünü toplamakta kullanılan kaşık & grass mowed and cured for use as fodder \\
    \cmidrule(rl){1-2}
    bir şeyi yapmayı önceden isteyip düşünme maksat & an act of intending a volition that you intend to carry out \\
    \cmidrule(rl){1-2}
    i̇pekböceği kozaları çözülerek çıkarılan ve dokumacılıkta kullanılan çok ince esnek ve parlak tel & a very strong thick rope made of twisted hemp or steel wire \\
    \cmidrule(rl){1-2}
    anadolu’nun doğu ve kuzey bölgelerinde en çok erzurum yöresinde el ele tutuşarak oynanan bir oyun & game a players turn to take some action permitted by the rules of the game \\
    \cmidrule(rl){1-2}
    başkalarından saklanan duyurulmayan saklı kalan mahrem & a consecrated place where sacred objects are kept \\
    \cmidrule(rl){1-2}
    bir kimseye ya da bir şeye karşı belli tavır takınmak & a feeling of sympathy for someone or something \\
    \cmidrule(rl){1-2}
    tarihsel gelişme içinde belirli bir üretim ilişkisinin belirli bir aşamasında bir arada yaşayan insanların tümü & a large number of things or people considered together \\
    \cmidrule(rl){1-2}
    bir cismin herhangi bir yöndeki uzanımı & a relatively small granular particle of a substance \\
    \cmidrule(rl){1-2}
    organizmada birtakım değişikliklerin ortaya çıkmasıyla fizyoloji görevlerinin bozulması durumu sayrılık maraz esenlik” karşıtı & an impairment of health or a condition of abnormal functioning \\
    \cmidrule(rl){1-2}
    yolcu ve gezmenlere geceleme olanağı sağlamak bunun yanında yemek eğlence gibi  türlü hizmetleri sunmak ereğiyle kurulmuş işletme & a building where travelers can pay for lodging and meals and other services \\
    \cmidrule(rl){1-2}
    hükümdar ailesinden olan erkeklere verilen san & female of any member of the dog family \\
    \cmidrule(rl){1-2}
    orduda yazı işleri ile uğraşan er ya da erbaş & a verbal or written request for assistance or employment or admission to a school \\
    \cmidrule(rl){1-2}
    düşmanın gelmesi beklenen yollar üzerinde askeri önem taşıyan kentlerde geçit ve darboğazlarda güvenliği sağlamak için yapılan kalın duvarlı  burçlu mazgallı yapı & an area within a building enclosed by walls and floor and ceiling \\
    \cmidrule(rl){1-2}
    i̇çi su ya da hava dolu ufak kabartı ya da kürecik & water in small drops in the atmosphere blown from waves or thrown up by a waterfall \\
    \cmidrule(rl){1-2}
    hava ya da herhangi bir akışkanı bir yerden başka bir yere basınç yoluyla aktarmaya yarayan makine & a vertical flue that provides a path through which smoke from a fire is carried away through the wall or roof of a building \\
    \cmidrule(rl){1-2}
    başkalarınca bilinmesi sakıncalı görülen bir gerçeği saklamaktan vazgeçip açıklama söyleme bildirme & an acknowledgment of the truth of something \\
    \cmidrule(rl){1-2}
    bir bilim sanat meslek dalıyla ya da bir konu ile ilgili özel ve belirli bir kavramı olan sözcük ıstılah & an occupation requiring special education especially in the liberal arts or sciences \\
    \cmidrule(rl){1-2}
    bir kilidi açıp kapamak için kullanılan araç açar açkı & a fastener fitted to a door or drawer to keep it firmly closed \\
    \cmidrule(rl){1-2}
    i̇şletilmek için bir yere ödünç verilen paraya karşılık alınan kâr ürem işlenti nema & a fixed charge for borrowing money usually a percentage of the amount borrowed \\
    \cmidrule(rl){1-2}
    yeryüzü parçası yerey yer toprak alan & sloping land especially the slope beside a body of water \\
    \cmidrule(rl){1-2}
    yiyecek koymaya yarar az derin ve yayvan kap & metal or earthenware cooking vessel that is usually round and deep often has a handle and lid \\
    \cmidrule(rl){1-2}
    yerleşik toplumsal düzeni köklü hızlı ve geniş kapsamlı olarak niteliksel değiştirme ve yeniden biçimlendirme eylemi inkılap & an extended social group having a distinctive cultural and economic organization \\
    \cmidrule(rl){1-2}
    duygusal olma durumu & an unstable situation of extreme danger or difficulty \\
    \cmidrule(rl){1-2}
    özel bir bozukluğu belirleyen bir arada görülen tanıyı kolaylaştıran bulgu ve  belirtilerin tümü & a pattern of symptoms indicative of some disease \\
    \cmidrule(rl){1-2}
    dumanı ocaktan çekip havaya vermeye yarayan maden ya da tuğla yol & a vertical flue that provides a path through which smoke from a fire is carried away through the wall or roof of a building \\
    \cmidrule(rl){1-2}
    bir canlının üstünü kaplayan ve onu dış etkilere karşı koruyan kendiliğinden oluşmuş sertçe bölüm kışır & the activity of protecting someone or something \\
    \cmidrule(rl){1-2}
    bir olayın ilk duyurusu olan biten salık & following the first in an ordering or series \\
    \cmidrule(rl){1-2}
    çoğunlukla kare ya da silindir biçimindeki yüksek yapı & a protective covering or structure \\
    \cmidrule(rl){1-2}
    bir yazıya başka bir yazarın yazısından alınmış parça aktarma iktibas & the form in which a text especially a printed book is published \\
    \cmidrule(rl){1-2}
    felsefeyle uğraşan ve felsefenin gelişmesine katkıda bulunan kimse felsefeci feylesof & a specialist in philosophy \\
    \cmidrule(rl){1-2}
    bir vücudun ya da bir organın yapı öğelerinden birini oluşturan gözeler bütünü nesiç & part of an organism consisting of an aggregate of cells having a similar structure and function \\
    \cmidrule(rl){1-2}
    fiziksel güç takat & the property of lacking physical or mental strength liability to failure under pressure or stress or strain \\
    \cmidrule(rl){1-2}
    i̇lgisini çekmek önem vermek ya da bir şeyle ilgili kılmak & an act of help or assistance \\
    \cmidrule(rl){1-2}
    askeri olmayan & a nonmilitary citizen \\
    \cmidrule(rl){1-2}
    bir olayı gören izleyen kimse izleyici & someone who takes part in an activity \\
    \cmidrule(rl){1-2}
    yapıları ve ulaşım araçlarını tren vapur gibi aydınlatmak havalandırmak amacıyla yapılan çerçeve cam panjur perde gibi eklentilerle daha kullanışlı  bir duruma getirilen açıklık & a vertical flue that provides a path through which smoke from a fire is carried away through the wall or roof of a building \\
    \cmidrule(rl){1-2}
    i̇nsanda ve omurgalılarda içinde beyin bulunan başın kemik bölümü & the bony skeleton of the head of vertebrates \\
    \cmidrule(rl){1-2}
    özellikle sokakta ayağı korumak için giyilen iskarpin çizme kundura mokasen sandalet patik galoş gibi türleri olan ayak giyeceği pabuç & a shoe consisting of a sole fastened by straps to the foot \\
    \cmidrule(rl){1-2}
    yüzeyi belirli uzunlukta bırakılmış hammadde lifleriyle kaplı parlak yumuşak kumaş & fabric woven from cotton fibers \\
    \cmidrule(rl){1-2}
    delik yırtık ya da eski bir yeri uygun bir parça ile onarma kapatma & a piece of cloth used as decoration or to mend or cover a hole \\
    \cmidrule(rl){1-2}
    ciltli ya da ciltsiz olarak bir araya getirilmiş basılı ya da yazılı kâğıt yaprakların tümü & one side of one leaf of a book or magazine or newspaper or letter etc or the written or pictorial matter it contains \\
    \cmidrule(rl){1-2}
    kuvvetin bir cismi bir nokta ya da bir eksen yöresinde döndürme etkisini  belirleyen vektör niceliği & the effect of one thing or person on another \\
    \cmidrule(rl){1-2}
    bir evin en geniş bölümü & an outbuilding or part of a building for housing automobiles \\
    \cmidrule(rl){1-2}
    bez tahta kâğıt gibi maddeler üzerine yapılmış yağlıboya suluboya pastel ya da karakalem resim & a three-dimensional work of plastic art \\
    \cmidrule(rl){1-2}
    i̇lgisiz olma durumu aldırmazlık alakasızlık & an unstable situation of extreme danger or difficulty \\
    \cmidrule(rl){1-2}
    sevgi ve bağlılık duyulan & an absence of emotion or enthusiasm \\
    \cmidrule(rl){1-2}
    anahtar düğme gibi takılıp çıkarılabilen bir parça yardımıyla çalışan kimi zaman elektronik de olabilen kapatma aygıtı & electronic equipment consisting of a device providing access to a computer has a keyboard and display \\
    \cmidrule(rl){1-2}
    evrende ya da düşüncede yer alan “yok” karşıtı bu sözcük hep yüklem olarak kullanılır ve üçüncü kişilerde koşaç almayabilir belirten olması için sonuna olan ortacı getirilir & something that should remain hidden from others especially information that is not to be passed on \\
    \cmidrule(rl){1-2}
    bir yapının dışarıya doğru çıkmış çevresi duvar ya da parmaklıkla çevrili bölümü & a movable barrier in a fence or wall \\
    \cmidrule(rl){1-2}
    evin ya da herhangi bir yapının oturmak çalışmak yatmak gibi işlere yarayan banyo salon giriş vb dışında kalan bir ya da birden fazla çıkışı olan bölmesi göz & a structure taller than its diameter can stand alone or be attached to a larger building \\
    \cmidrule(rl){1-2}
    kurulanmaya yarar havlı bez & white goods or clothing made with linen cloth \\
    \cmidrule(rl){1-2}
    kamuyla ilgili işlerin yürütülmesi için gerekli gelirleri ve harcanan paraları düzenleyen kuralların tümü & the management of money and credit and banking and investments \\
    \cmidrule(rl){1-2}
    bir doğal su birikintisinin yanındaki alan kıyı & an area of sand sloping down to the water of a sea or lake \\
    \cmidrule(rl){1-2}
    tene yumuşaklık vermek ya da güneş yağmur gibi dış etkilerden korunmak için sürülen güzel kokulu merhem & fine powdery material such as dry earth or pollen that can be blown about in the air \\
    \cmidrule(rl){1-2}
    sınırlamak eylemi & a rule or condition that limits freedom \\
    \cmidrule(rl){1-2}
    bir sanata bir bilime temel olan yön veren ilke kaide & an occupation requiring special education especially in the liberal arts or sciences \\
    \cmidrule(rl){1-2}
    i̇l ilçe gibi yerleşim bölgelerinde iki yanında evler olan caddeye oranla  daha dar ya da kısa olabilen yol & a deep and relatively narrow body of water as in a river or a harbor or a strait linking two larger bodies that allows the best passage for vessels \\
    \cmidrule(rl){1-2}
    belli bir saatte belli bir yerde iki ya da daha çok kişi arasında kararlaştırılan  buluşma & a time period usually extending from friday night through sunday more loosely defined as any period of successive days including one and only one sunday \\
    \cmidrule(rl){1-2}
    araştırılıp öğrenilmesi düşünülüp çözümlenmesi bir sonuca bağlanması gereken durum mesele problem & a question raised for consideration or solution \\
    \cmidrule(rl){1-2}
    alışılmış olandan umulandan ya da gerekenden eksik niceliği küçük “çok” karşıtı & the quality of having an inferior or less favorable position \\
    \cmidrule(rl){1-2}
    herhangi bir iş bir görev için kendini ileri sürme ya da başkaları tarafından ileri sürülme namzetlik & the real physical matter of which a person or thing consists \\
    \cmidrule(rl){1-2}
    birini telefonla aramak ve bir şey söylemek & a seat for one person with a support for the back \\
    \cmidrule(rl){1-2}
    denemek eylemi sınama tecrübe & the act of rejecting something \\
    \cmidrule(rl){1-2}
    patlıcangillerden yaprakları tüylü çiçekleri salkım durumunda vitamince zengin kırmızı ürünü için yetiştirilen bir bitki lycopersicon esculentum & mildly acid red or yellow pulpy fruit eaten as a vegetable \\
    \cmidrule(rl){1-2}
    bir görevi bir işi yasaların verdiği olanaklara göre belli koşullarla yürütmeyi sağlayan hak salahiyet mezuniyet & financial aid provided to a student on the basis of academic merit \\
    \cmidrule(rl){1-2}
    bir konu ile ilgili bilgi vermek ve bu bilgiler üzerinde tartışmak amacıyla birkaç yetkilinin yönetimi altında düzenlenen toplantı & a prearranged meeting for consultation or exchange of information or discussion especially one with a formal agenda \\
    \cmidrule(rl){1-2}
    kendi isteğiyle görevden ayrılma çekilme işinden ayrılma & withdrawal from your position or occupation \\
    \cmidrule(rl){1-2}
    merkezde bulunan ve bir eksenin çevresinde dönebilir kurs ya da çember teker & an urban area with a fixed boundary that is smaller than a city \\
    \cmidrule(rl){1-2}
    bireyle ilgili olan bireye özgü olan ferdi & a person who is a member of a partnership \\
    \cmidrule(rl){1-2}
    yumuşakçalardan bahçelerde yaşayan sarmal kabuklu küçük hayvan helix & elongated crescent-shaped yellow fruit with soft sweet flesh \\
    \cmidrule(rl){1-2}
    kadın ya da  erkek çocuğun en ince sesi & the highest female voice the voice of a boy before puberty \\
    \cmidrule(rl){1-2}
    omuz başının altında kolun gövde ile birleştiği yer & the part of the body between the neck and the upper arm \\
    \cmidrule(rl){1-2}
    üzerinde sıcak ve soğuk su muslukları bulunan porselen sac emaye gibi maddelerden yapılan el yüz bulaşık yıkamaya yarar yer & metal or earthenware cooking vessel that is usually round and deep often has a handle and lid \\
    \cmidrule(rl){1-2}
    adaletle iş gören adaletten haktan ayrılmayan hakkı yerine getiren adaletli & a rule or condition that limits freedom \\
    \cmidrule(rl){1-2}
    tanıtma filmi & gathering of producers to promote business \\
    \cmidrule(rl){1-2}
    bir makinenin herhangi bir taşıtın hızını kesmeye ya da onu durdurmaya yarayan düzenek & a restraint used to slow or stop a vehicle \\
    \cmidrule(rl){1-2}
    ara uzaklık & a large distance \\
    \cmidrule(rl){1-2}
    kap kılıf sarma & small thin inflatable rubber bag with narrow neck \\
    \cmidrule(rl){1-2}
    herhangi birinden & the effect of one thing or person on another \\
    \cmidrule(rl){1-2}
    sinema ya da müzikhol sanatçısı yıldız & a three-dimensional work of plastic art \\
    \cmidrule(rl){1-2}
    okuyup yazmadan başlayarak en yüksek düzeyde bilim ve sanat bilgisi vermeye değin çeşitli derecede toplu olarak öğrenimin sağlandığı yer mektep & an occupation requiring special education especially in the liberal arts or sciences \\
    \cmidrule(rl){1-2}
    başın altına koymak ya da sırtı dayamak için kullanılan içi yün pamuk kuştüyü gibi şeylerle doldurulmuş küçük minder & a piece of cloth used as decoration or to mend or cover a hole \\
    \cmidrule(rl){1-2}
    alt ve üst tabanları birbirine eşit dairelerden oluşan bir nesnenin eksenini dikey olarak kesen birbirine koşut iki yüzeyin sınırladığı cisim  üstüvane & a support that consists of a horizontal surface for holding objects \\
    \cmidrule(rl){1-2}
    bir kimsenin belli bir sürede ya da yaşam boyu edindiği bilgilerin tümü tecrübe & the real physical matter of which a person or thing consists \\
    \cmidrule(rl){1-2}
    yaşantıları öğrenilen konuları bunların geçmişle ilişkisini bilinçli olarak saklama gücü hafıza & a storage device on which information sounds or images have been recorded \\
    \cmidrule(rl){1-2}
    geçmeye engel olacak biçimde uzunlamasına kazılmış derin çukur & the general feeling that some desire will be fulfilled \\
    \cmidrule(rl){1-2}
    bıçak makas gibi bir araçla bir şeyi ikiye ayırmak & a top as for a bottle \\
    \cmidrule(rl){1-2}
    yitme yitim & a feeling of restless agitation \\
    \cmidrule(rl){1-2}
    eski çağlardan beri söylenegelen olağanüstü varlıkları olayları konu edinen imgesel öykü söylence & the series of events that form a plot
\end{longtable}

%\include{Appendices/AppendixB}
%\include{Appendices/AppendixC}

%----------------------------------------------------------------------------------------
%	BIBLIOGRAPHY
%----------------------------------------------------------------------------------------

\printbibliography[heading=bibintoc]

%----------------------------------------------------------------------------------------

\end{document}
